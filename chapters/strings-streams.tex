As part of solving various assignments,
you will frequently need to read and process strings,
and input and output in C++ are typically handled using streams.
While you should have learned these in your introductory programing
class, we will give a brief review here, focusing on issues that seem
to often cause students difficulties.
You can look up exact syntax in your introductory C++ textbook or
online.

\section{Strings}
In basic C, string are implemented just as arrays of characters. 
There is no separate string type. 
This means that you need to allocate enough space for your character
array to hold any input string that you might receive.
It also means that you need a special way of marking how much of that
array you are \emph{actually} using: just because you allocated, say,
80 characters in your string does not mean you will always use all 80.
The way C does this is by having a special role for the character
\code{\textbackslash{}0}.
(That's the character number 0, not the actual character `0'.)
It marks the end of the used part of the string.

The \code{string} class in C++ is a little easier to use:
in particular, it hides some of the memory management for you,
so that you don't have to worry about allocating enough space.
It also implements some operators that you will frequently use, such
as \code{=} (assignment), \code{==} (comparison) and \code{+}
(appending strings). 
The \code{string} class also contains a number of other very useful
functions. Since you will be doing a lot of processing of data you are
reading in from files in this class, you can probably save yourself a
lot of work by reviewing the \code{string} class and its features.

One string function which is really useful in C,
but to our knowledge does not have an equivalent in C++ in standard
implementations, is \code{strtok}.
It allows you to tokenize a string, i.e., break it into smaller chunks
using any set of characters as separators.
Check it out!

\section{Streams}
Streams are the C++ ways of interacting with files, keyboard, screen,
and also with strings.
There are streams you read (keyboard, files, strings)
and streams you write (screen, files, strings).
In all cases, they are basically a sequence of characters
that you are extracting from (read) or inserting into (write).
They are a fairly clean abstraction of reading and writing ``items''
one by one.

The standard streams you will be interacting with are given in
Table~\ref{tab:streams}. 
They require you to \code{\#{}include} different header files, also
listed here.

\begin{table}[htb]
\begin{tabular}{l|l|l}
Stream & Target & Header File \\ \hline
\code{cin} & Keyboard & \code{iostream} \\
\code{cout}, \code{cerr} & Screen & \code{iostream} \\
\code{ifstream} & File to read & \code{fstream} \\
\code{ofstream} & File to write & \code{fstream} \\
\code{stringstream} & String & \code{sstream} \\ \hline
\end{tabular}
\caption{The streams you will use most frequently.\label{tab:streams}}
\end{table}

The difference between \code{cout} and \code{cerr} is twofold:
First, \code{cout} buffers its output instead of printing it immediately,
whereas \code{cerr} prints immediately. 
This makes \code{cerr} very useful to output debug information,
whereas you will probably write your ``real'' output to \code{cout}.
They are also two different logical streams. This means that if you
redirect your output to a file (which we will do when grading you),
the default is that only \code{cout} gets redirected.
So if you print debug information to \code{cerr},
it still appears on the screen, and doesn't ``pollute'' your output. 
Finally, \code{cerr} often gets displayed in a different color.
All this is to say: it may be good to get into the habit to print all
your ``real'' output to \code{cout},
and all of your debug output to \code{cerr}.

To extract items from an input stream, you use the \code{>>} operator,
and to insert items into an output stream, you use \code{<<}.

A few things are worth remembering about reading from an input
stream. (These frequently trip students up.)
\begin{itemize}
\item When reading from a stream (say, \code{cin}, or a \code{ifstream
    myinput}), you can use \code{cin.fail()} (or \code{myinput.fail()})
  to check whether the read succeeded.
  When one read fails (and the \code{fail} flag is set), it will
  remain set until you explicitly reset it with \code{cin.clear()} or
  \code{myinput.clear()}. Until then, all further reads will fail,
  since C++ assumes that until you have fixed the source of the problem,
  all incoming data will be unreliable now.

\item Remember that \code{>>} reads until the next white space, which
  includes spaces, tabs, and newlines.

  Be careful here when entering things manually. If at the first
  prompt, you enter multiple items separated by space, they will
  ``supply'' the next few extractions. That is, if you write code like
\begin{verbatim}
cout << "Please enter n:"
cin >> n;
cout << "Please enter m:"
cin >> m;
\end{verbatim}
  and enter ``5 3'' at the first prompt, it will read 5 into \code{n}
  and 3 into \code{m}.
  In particular, this is likely to trip you up if you somehow try to
  read an entire line of text from a file using \code{>>}, and that
  line happens to contain white spaces.

\item If you want to also read spaces, then instead of using
  \code{>>}, you should use \code{getline}, such as
  \code{cin.getline()} or \code{myinput.getline()}. 
  It lets you specify a string to read into, 
  a maximum number of characters to read, and 
  character to stop at (default is newline), and will then read into
  the string. That string can later be further processed.

  Notice that you need to be a bit careful about \code{getline} at the
  end of a file; you may need to check the correct flags to avoid reading the
  last line multiple times.

\item If you want to ``clear'' off the remaining characters in a
  stream so they don't cause issues later, you can use the
  \code{ignore (int n)} function, which will ignore the next \code{n}
  characters (which you can specify --- default is 1).
\end{itemize}

\subsection*{File Streams and String Streams}
File streams are the typical C++ way of reading data from files.
The two types are \code{ifstream} (for files you read) and
\code{ofstream} (for files you write).
A file stream needs to be opened before you can read/write from it. 
Once you have opened it, it can basically be used like \code{cin} and
\code{cout}.

Notice that in file streams, when you read \emph{over} the end of the
file, you will get a \code{fail} condition.
That happens after you \emph{reach} the end of the file,
which is often cause for confusion when using \code{getline}.

The other thing to keep in mind about file streams is that after
you are done, you should call \code{close()} on them.

\medskip

Since strings are sequences of characters, just like files,
it is natural to treat them as streams as well.
This is what the \code{stringstream} type is for.
It is particularly useful when you have read an entire string and want
to break it into smaller pieces, e.g., separated by spaces.
You can treat the string as a \code{stringstream}, and then use extraction
operations on it to get the different items.
This is also a convenient way to convert strings to integers.

As a piece of advice, avoid using the same \code{stringstream} object
multiple times --- or if you do, you must make sure to reset it.