\newcommand{\ProblemTree}{%
\pstree{\Tcircle[fillstyle=solid,fillcolor=black]{\phantom{5}}}
	{
   	   \Ttri{$T$}
	}
}

In the next few chapters, we will learn of the two typical ways of
implementing maps: Search Trees and Hashtables.
Remember that in a map, we store associations between keys and values.
We need to be able to insert and remove such associations, and to look
up the associated value for a given key.
We already saw a brief preview of Search Trees in Section~\ref{sec:trees.search-trees}.
There, we only showed the keys in our graphic depiction.
In this and the next chapter, we will do the same thing. 
However, even though we only show keys in our diagrams, it is implied
that for each key, there is also an associated value stored, and that
looking up this value is really the main reason for studying trees in
the first place.

Also, for maps, one typically assumes that there can only be one copy
of each key. If you explicitly want multiple copies of the same key,
you would call it a multi-map. It makes some implementation details
more messy, and we want to focus on the essentials here. So from now
on, in all of our examples, we will not have the possibility of
duplicates among the keys.

\section{The Basics}
While many search trees (including our introductory example in
Section~\ref{sec:trees.search-trees})
are binary, the ones we will analyze here in depth are not.
2-3 Trees differ from binary search trees in that each node contains 1
or 2 keys.
Unless it is a leaf, a node with 1 key has exactly 2 children, and a
node with 2 keys has exactly 3 children.
This can be generalized further: we can build trees with $m$ keys in
which each non-leaf node has $m+1$ children. We will see 2-3-4 trees
later (where nodes have 1--3 keys), and in practice, there are reasons
to use trees in which nodes have several hundred or thousand keys.

Also, we will see later that as a temporary step in adding a key to a
2-3 Tree, we sometimes have nodes with 3 keys and 4 subtrees;
similarly, as part of deleting a key, we sometimes end up with a node
with 0 keys and one subtree. 

We'll see those exceptions soon enough.
The defining properties of 2-3 Trees are the following:

\begin{itemize}
\item All leaves must be at the same level.
\item For nodes with 1 key $k$, the binary search tree property holds:
  all keys in the left subtree are smaller than $k$, 
  and all keys in the right subtree are greater than $k$.
\item For nodes with 2 keys $k_1 < k_2$, the three subtrees $T_1, T_2,
  T_3$ satisfy the following property:
  \begin{enumerate}
  \item All keys in $T_1$ are smaller than $k_1$.
  \item All keys in $T_2$ are strictly between $k_1$ and $k_2$.
  \item All keys in $T_3$ are greater than $k_2$.
  \end{enumerate}
\end{itemize}

It will be our job in building these trees to ensure that the defining
properties hold at all times. We will see later how to do that.
We will also see how to do search, and why the depth of the tree is
$O(\log n)$.

To implement a 2-3 Tree, we will need a slightly revised version of a
node, to accommodate multiple keys and values. The class would look as
follows:

\begin{verbatim}
template <class KeyType, class ValueType>
class Node {   // variables could be public, or could add getters/setters
  int numberKeys;
  KeyType key[2];      // space for up to 2 keys
  ValueType value[2];  // space for the corresponding values
  Node* subtree[3];    // pointers to the up to 3 subtrees
}
\end{verbatim}

Internally, the best way to store a 2-3 Tree is probably with a
dynamic memory structure and pointers, storing a pointer to the root
somewhere. Initially, when the tree is empty, we can set the root
pointer to \code{NULL}. The tree will grow both in number of nodes and
height as insertions happen, so obviously, we do not want to fix its
size ahead of time.

We could also try to store the tree in an array (like we did for
Priority Queues), but lose all of the advantages of doing so, as index
calculations are not possible any more. When the tree was complete and
binary, it was very easy to figure out which were the children of a
given node, but here, it would depend on the degrees of many other
nodes, which would take way too long to look up.

\section{Search and Traversal}
The first useful observation about 2-3 Trees is that --- like other
search trees --- they make it really easy to search for a key, and to 
output all keys in sorted order. 

To search for a key, we simply generalize the search procedure for
binary trees. 
When a particular \code{key} is requested, and we search in a subtree
with a given root \code{r}, we compare \code{key}
to \code{r->key[0]} (and \code{r->key[1]}, if \code{r} has two keys).
If we found the key, great. Otherwise, we can determine which of the
two (or three) subtrees to recursively search in.
The code would look as follows:

\begin{verbatim}
ValueType search (KeyType key, Node *r) const
{
  if (r == NULL) throw KeyNotFoundException;
  else {
     if (key < r->key[0]) return search (key, r->subtree[0]);
     else if (key == r->key[0]) return r->value[0];
     else if (numberKeys == 1 || key < r->key[1]) return search (key, r->subtree[1]);
     else if (key == r->key[1]) return r->value[1];
     else return search (key, r->subtree[2]);
  }
}
\end{verbatim}

To traverse all of the keys in sorted order, 
we simply use depth-first in-order traversal (see Section~\ref{sec:trees.traversing}).
That is, at any node, we first recursively visit the entire
left subtree, then the key (or left key, if there are two), then the
middle tree (if it exists), the right key (if it exists), and finally
the right subtree. This is the natural generalization of in-order
traversal from binary trees to 2-3 Trees.

This describes how to implement an ordered iterator over search
trees. This is one of the main advantages of search trees over
Hashtables as data structures for maps. For Hashtables, it's
much harder to implement an efficient Iterator. 
(Hashtables have other advantages to compensate.)

\section{Inserting and Removing Keys: Overview}
In implementing insertions and deletions on 2-3 Trees, we should keep
in mind the following:

\begin{itemize}
\item The 2-3 Tree properties must always hold after we are done with
  the operation, so that the tree remains balanced. 
  Thus, we must ensure that all leaves are at the same level,
  two-key nodes have three children, 
  one-key nodes have two children, 
  and we can't have any number other than 1 or 2 keys in one node. 
\item All insertions and removals in search trees typically (and
  definitely for 2-3 Trees) happen at a leaf; that's much easier to
  deal with than changing things at internal nodes.
\item Typically, pre-processing (for removals) and/or post-processing
  (to restore various properties) is required, in addition to just
  adding or removing keys.

  More concretely, we saw for priority queues that the way to insert
  or remove keys was to temporarily violate the heap property, and
  then fix it with a recursive function. 
  For 2-3 Trees, we will do the same thing.

  In doing this, we will \emph{never, ever} violate the first property
  (having all leaves at the same level); this property would be
  incredibly hard to restore. 
  Instead, we will temporarily generate nodes with 0 keys (for
  deletion) and with 3 keys (for addition), and then fix that systematically.
\end{itemize}


In explaining how insertions and removals work, our running example
will be the tree $T$ shown in Figure~\ref{fig:running-example}:

\begin{figure}[htb]
\begin{center}
\psset{unit=0.7cm,levelsep=7ex,arrowsize=0.1 3}
\begin{pspicture}(0,-4)(0,1)
\pstree{\Toval{6 14}}
        {
	 \pstree{\Tcircle{4}}
                {
		  \Toval{1 3}
		  \Tcircle{5}
                }
	 \pstree{\Toval{8 10}}
                {
		  \Tcircle{7}
		  \Tcircle{9}
		  \Toval{11 13}
                }
	 \pstree{\Tcircle{19}}
                {
		  \Tcircle{17}
		  \Tcircle{23}
                }
        }
\end{pspicture}
\caption{Our running example $T$ of a 2-3 Tree.\label{fig:running-example}}
\end{center}
\end{figure}

\section{Inserting Keys}
Whenever we insert a key (or, more precisely, a key-value pair) into a
tree, the first thing we do is search for it, unsuccessfully, of
course. (If we found it, that would be a duplicate, so we'd throw an
exception or otherwise signal that the key was already in the map.)
We then insert it into the leaf node where the search
terminated. If the leaf node only had one key before, it has two keys
now, and everything is good.
But if the leaf node already had two keys, then it now has three keys,
so we'll need to fix it.
We will see how to fix it by looking at the effects of inserting a
number of keys into our example tree $T$.

First, let's look at the case of inserting 21 into $T$.
To do so, we search for 21. When we do that, we will terminate at the
node that already contains the key 23.
This leaf node has only one key, so we can simply add 21.
The result will only be a local change, and our tree now looks as
in Figure~\ref{fig:insert-21}:

\begin{figure}[htb]
\begin{center}
\psset{unit=0.7cm,levelsep=7ex,arrowsize=0.1 3}
\begin{pspicture}(0,-4)(0,1)
\pstree{\Toval{6 14}}
       {
	  \pstree{\Tp}{\Ttri{\phantom{5}}}
	  \pstree{\Tp}{\Ttri{\phantom{5}}}
          \pstree{\Tcircle{19}}
		{
		  \Tcircle{17}
		  \Toval{21 23}
		}
       }
\end{pspicture}
\caption{$T$ after inserting the key 21.
In this and subsequent figures, we'll just write triangles to denote
parts that are unchanged or not particularly interesting.\label{fig:insert-21}}
\end{center}
\end{figure}
	
This was easy. Of course, we won't always be so lucky as to insert
into a node with only a single key. So next, let's look at what
happens when we insert 26 into the tree we got after inserting 21.
The key 26 also belongs in the node with 21 and 23, so when we add it
to the node, that leaf now contains three keys, as shown in Figure~\ref{fig:insert-26-1}:

\begin{figure}[htb]
\begin{center}
\psset{unit=0.7cm,levelsep=7ex,arrowsize=0.1 3}
\begin{pspicture}(0,-4)(0,1)
\pstree{\Toval{6 14}}
       {
	  \pstree{\Tp}{\Ttri{\phantom{5}}}
	  \pstree{\Tp}{\Ttri{\phantom{5}}}
          \pstree{\Tcircle{19}}
		{
		  \Tcircle{17}
		  \Toval{21 23 26}
		}
       }
\end{pspicture}
\caption{After inserting the key 26.
Having three keys in a node is illegal, so we need to fix this.\label{fig:insert-26-1}}
\end{center}
\end{figure}

To fix this, we can break up a node with 3 keys, by moving its middle
key up to the parent node, and making the two other keys into separate
nodes with one key each.
In our example, 23 gets moved into the node that currently contains
only the key 19, turning it into a node containing 19 and 23. 
The other two elements, 21 and 26, are split into two separate nodes
(since the  parent node is now a two-key node, it can support three
children), and these nodes become the children nodes for the node
containing 19 and 23.
The resulting tree looks as shown in Figure~\ref{fig:insert-26-2}.
	
\begin{figure}[htb]
\begin{center}
\psset{unit=0.7cm,levelsep=7ex,arrowsize=0.1 3}
\begin{pspicture}(0,-4)(0,1)
\pstree{\Toval{6 14}}
       {
	  \pstree{\Tp}{\Ttri{\phantom{5}}}
	  \pstree{\Tp}{\Ttri{\phantom{5}}}
	  \pstree{\Toval{19 23}}
		{
			\Tcircle{17}
			\Tcircle{21}
			\Tcircle{26}
		}
       }
\end{pspicture}
\caption{After fixing the problem from inserting the key 26.\label{fig:insert-26-2}}
\end{center}
\end{figure}

So in this case, with just one split operation, the problem was fixed.
Next, let's see what happens when we insert 12 into $T$.
(So we're thinking of inserting it into the \emph{original} tree, but
it makes no difference in this particular case if you want to think of
inserting it into the tree after we already inserted 21 and 26.)

Again, we first search for 12, ending up in the leaf that contains the
keys 11 and 13 already. Again, we add it to the leaf node, resulting
in 3 keys in that node. 
The resulting situation is shown in Figure~\ref{fig:insert-12-1}. 

\begin{figure}[htb]
\begin{center}
\psset{unit=0.7cm,levelsep=7ex,arrowsize=0.1 3}
\begin{pspicture}(0,-4)(0,1)
\pstree{\Toval{6 14}}
        {
	 \pstree{\Tp}{\Ttri{\phantom{5}}}
	 \pstree{\Toval{8 10}}
                {
		  \Tcircle{7}
		  \Tcircle{9}
		  \Toval{11 12 13}
                }
         \pstree{\Tp}{\Ttri{\phantom{5}}}
        }
\end{pspicture}
\caption{The result of inserting the key 12 into $T$.\label{fig:insert-12-1}}
\end{center}
\end{figure}

As before, we take the middle key --- here, 12 --- and move it up to
the parent's node, splitting the keys 11 and 13 into separate new
nodes. But now, the parent has 3 keys, and 4 subtrees, as shown in
Figure~\ref{fig:insert-12-2}. So we'll have to fix that.

\begin{figure}[htb]
\begin{center}
\psset{unit=0.7cm,levelsep=7ex,arrowsize=0.1 3}
\begin{pspicture}(0,-4)(0,1)
\pstree{\Toval{6 14}}
        {
	 \pstree{\Tp}{\Ttri{\phantom{5}}}
	 \pstree{\Toval{8 10 12}}
                {
		  \Tcircle{7}
		  \Tcircle{9}
		  \Tcircle{11}
                  \Tcircle{13}
                }
         \pstree{\Tp}{\Ttri{\phantom{5}}}
        }
\end{pspicture}
\caption{Moving the key 12 one level up.\label{fig:insert-12-2}}
\end{center}
\end{figure}

To fix this new problem of a node with 3 keys, we again take the
middle key --- in this case, 10 --- and move it up one level, into the
parent node with keys 6 and 14.
Now, the keys 8 and 12 get split and become their own nodes; the new
node containing 8 obtains the left two subtrees (here, nodes with keys
7 and 9), while the node with key 12 obtains the right two subtrees,
with keys 11 and 13. 
The resulting tree is shown in Figure~\ref{fig:insert-12-3}.

\begin{figure}[htb]
\begin{center}
\psset{unit=0.7cm,levelsep=7ex,arrowsize=0.1 3}
\begin{pspicture}(0,-4)(0,1)
\pstree{\Toval{6 10 14}}
        {
	 \pstree{\Tp}{\Ttri{\phantom{5}}}
	 \pstree{\Tcircle{8}}
                {
		  \Tcircle{7}
		  \Tcircle{9}
                }
         \pstree{\Tcircle{12}}
                {
		  \Tcircle{11}
                  \Tcircle{13}
                }
         \pstree{\Tp}{\Ttri{\phantom{5}}}
        }
\end{pspicture}
\caption{Moving the key 10 one level up.\label{fig:insert-12-3}}
\end{center}
\end{figure}
	
Now, we have moved the problem yet another level up. 
The root node now contains three keys 6, 10, and 14.
So we do the same thing: we push up the middle key (here, 10) one
layer, and split the other two keys into separate nodes, each
inheriting two of the subtrees. 
The result is shown in Figure~\ref{fig:insert-12-4}

\begin{figure}[htb]
\begin{center}
\psset{unit=0.7cm,levelsep=7ex,arrowsize=0.1 3}
\begin{pspicture}(0,-5)(0,1)
\pstree{\Tcircle{10}}
       {
        \pstree{\Tcircle{6}}
               {
                \pstree{\Tp}{\Ttri{\phantom{5}}}
                \pstree{\Tcircle{8}}
                        {
                        \Tcircle{7}
                        \Tcircle{9}
                        }
                }
         \pstree{\Tcircle{14}}
                {
                 \pstree{\Tcircle{12}}
                        {
                        \Tcircle{11}
                        \Tcircle{13}
                        }
                 \pstree{\Tp}{\Ttri{\phantom{5}}}
                }
        }
\end{pspicture}
\caption{Moving the key 10 one more level up.\label{fig:insert-12-4}}
\end{center}
\end{figure}

At this point, we finally have obtained a legal 2-3 tree again, and
the process terminates. Notice that we have now increased the height
of the tree by one. Some students were wondering in class how the tree
can ever grow in height if we always insert at leaves. Now we know. It
grows at the root, even though insertion happens at the leaves.

The important thing to keep in mind is that whenever a node has too
many keys, you move up the \emph{middle} key to the parent node, and
split the remaining two keys into two separate new nodes. The left
node inherits the two left subtrees, while the right node inherits the
two right subtrees.

Below is pseudocode to help you understand how to insert into 2-3
Trees in general.
\begin{verbatim}
Inserting a new key
	
1. Search for the key. If found, deal with duplicate.
   Otherwise, we have found the leaf where the key belongs.
2. Insert the key into the leaf.
      If the leaf only had one key before, we are done now.
3. Otherwise, we must now fix the leaf.

To fix a node with 3 keys (a b c):
1. Move b up into the parent node if there is one.
   (Otherwise, create a new root for the tree, containing just b.)
2. Create a node with just a as the key, and make it the left child of the b node.
3. Create a node with just c as the key, and make it the right child of the b node.
\end{verbatim}


Notice that insertion as we defined it maintains the properties of a
2-3 Tree: the number of levels on all subtrees is increased by one
simultaneously when creating a new root, and not increased for any node
otherwise. 

Let's try to analyze the running time. To find the right leaf node
takes $\Theta(h)$, where $h$ is the height of the tree.
To actually insert the key into the tree takes $\Theta(1)$.
Breaking any three-key node into new smaller nodes takes $\Theta(1)$,
and we have to do this at most once for each level of the tree, so it
takes $\Theta(h)$.

So next, we want to calculate the height of a 2-3 Tree. We do this the
way we have calculated tree heights before.
If the tree has height $h$, because all leaves are at the same level,
and each internal node has at least 2 children, it has at least as
many nodes as a complete binary tree of $h$ levels (plus one more),
giving us an upper bound of $\log_2(n)$ on the height, as we
calculated before in the context of heaps.
On the other hand, a tree of $h$ levels with maximum degree 3 has at
most 1 node on level 0, 3 nodes on level 1, 9 nodes on level 2, and
generally $3^k$ nodes on level $k$, for a total of at most
$\sum_{k=0}^{h-1} 3^k = \frac{3^h-1}{3-1} \leq \frac{3^h}{2}$ nodes.
Thus, $n \leq 3^h/2$, giving us that $h \geq \log_3(2n) \geq \log_3(n)$.
Thus, $h$ is between $\log_3(n)$ and $\log_2(n)$, and those two are
within small constant factors of each other.
Thus, the runtime for insertion is $\Theta(\log (n))$.
(We can ignore the base of the logarithm, as they are off only by
constant factors.)

\section{Removing Keys}
To delete a key (and its value) from a 2-3 tree, we of course first
have to find it. Once we have found the key, deleting it from an
internal node would be really complicated. Therefore, in 2-3 trees
(and pretty much all search trees), we always have a first step of
ensuring that we only delete keys from leaf nodes.

The way to do this is to search for the next-larger (or possibly
next-smaller, but for concreteness, we'll stick with next-larger)
key. This is accomplished by first going down the edge immediately to
the right of the key to delete. (That is, the middle edge if the key
was the left key, or the right edge if the key was the right key.)
Then, we always go down the leftmost edge of all nodes, until we hit a
leaf. At that point, the leftmost key in that leaf is the successor.

Once we have found the successor, we swap it with the key we wanted to
delete. This is safe to do, as no other element could now be out of
place --- everything in the right subtrees is larger than the
successor. Now, the key we want to delete is in a leaf, and we can
start from there.

The deletion algorithm will start with deleting a key at a leaf, but
then possibly propagate necessary ``fixes'' up the tree.
By far the easiest case is if the leaf contained two keys. 
Then, we simply delete the key, and we are done.

Otherwise, we now have a leaf with no keys in it. We can't just delete
the leaf, as this would violate the 2-3 tree property of all leaves
being at the same level. So we need to find ways to fix the tree.
The operations we perform will be the same at the leaf level as at the
higher levels, so we'll phrase the problem case more generally.

At the point when we need to fix something, we have an internal node
with 0 keys and one subtree hanging off. In the base case, this
subtree is just \code{NULL}, but as we propagate up (recursively or
iteratively), there will be other subtrees. Thus, the problem
situation looks as shown in Figure~\ref{fig:problem-tree}.
(The black node has no keys, and one subtree below it, which is empty
for a leaf.)

\begin{figure}[htb]
\begin{center}
\psset{unit=0.7cm,levelsep=7ex,arrowsize=0.1 3}
\begin{pspicture}(-3,-2)(3,1)
\ProblemTree
\end{pspicture}
\caption{The problem case in deletion. The black node has no keys. \label{fig:problem-tree}}
\end{center}
\end{figure}

In order to deal with this problem case, we distinguish a few cases,
based on how many neighbors the empty node has, and how many keys they
in turn have.

\begin{enumerate}
\item If the empty node has a sibling with two keys, then the node can
  borrow a key from a sibling. A typical version of this case would
  look as shown in Figure~\ref{fig:borrow-from-neighbor}:

\begin{figure}[htb]
\begin{center}
\psset{unit=0.7cm,levelsep=7ex,arrowsize=0.1 3}
\begin{pspicture}(-3,-5)(3,1)
\pstree{\Tcircle{9}}
       {
        \pstree{\Toval{4 7}}
               {
                 \pstree{\Tp}{\Ttri{$T_1$}}
                 \pstree{\Tp}{\Ttri{$T_2$}}
                 \pstree{\Tp}{\Ttri{$T_3$}}
               }
        \ProblemTree
       }

\pstree{\Tcircle{7}}
      {
        \pstree{\Tcircle{4}}
               {
                 \pstree{\Tp}{\Ttri{$T_1$}}
                 \pstree{\Tp}{\Ttri{$T_2$}}
               }
        \pstree{\Tcircle[fillstyle=solid,fillcolor=lightgray]{9}}
               {
                 \pstree{\Tp}{\Ttri{$T_3$}}
                 \pstree{\Tp}{\Ttri{$T$}}
               }
       }
\end{pspicture}
\caption{Borrowing from a sibling. The empty node borrows from its
  sibling (via its parent) and is shown in gray after. \label{fig:borrow-from-neighbor}}
\end{center}
\end{figure}

It works similarly if the empty node has two siblings.
When the immediately adjacent sibling has two keys, we can do the same
thing (by pushing one of the sibling's keys to the parent, and the
parent's key down to the empty node).
When it is the sibling two hops away that has two keys, we push one
key up, a key from the parent down to the middle child, and the key
from the middle child up to the parent.

\item If no sibling of the empty node has two keys, then the next case
  is that the empty node has two siblings (i.e., the parent has two
  keys). In that case, the empty node can borrow a key from the
  parent, with help from the siblings.
  A typical case is shown in Figure~\ref{fig:borrow-from-parent}.

\begin{figure}[htb]
\begin{center}
\psset{unit=0.7cm,levelsep=7ex,arrowsize=0.1 3}
\begin{pspicture}(0,-5)(3,1)
\pstree{\Toval{4 9}}
       {
        \pstree{\Tcircle{2}}
               {
                 \pstree{\Tp}{\Ttri{$T_1$}}
                 \pstree{\Tp}{\Ttri{$T_2$}}
               }
        \pstree{\Tcircle{6}}
               {
                 \pstree{\Tp}{\Ttri{$T_3$}}
                 \pstree{\Tp}{\Ttri{$T_4$}}
               }
        \ProblemTree
       }

\pstree{\Tcircle{4}}
       {
        \pstree{\Toval{2}}
               {
                 \pstree{\Tp}{\Ttri{$T_1$}}
                 \pstree{\Tp}{\Ttri{$T_2$}}
               }
        \pstree{\Tcircle[fillstyle=solid,fillcolor=lightgray]{6 9}}
               {
                 \pstree{\Tp}{\Ttri{$T_3$}}
                 \pstree{\Tp}{\Ttri{$T_4$}}
                 \pstree{\Tp}{\Ttri{$T$}}
               }
       }
\end{pspicture}
\caption{Borrowing from the parent. The empty node's sibling borrows from its
  parent and is shown in gray after. \label{fig:borrow-from-parent}}
\end{center}
\end{figure}

This works pretty much the same way if instead of the right child, it
is the middle or left child of the parent that was empty.

\item The final remaining case is when the empty node has only one
  sibling, and that sibling contains only one key.
  In that case, we cannot fix the problem immediately. 
  Instead, we merge the parent and the sibling into one node with two
  keys, and push up the empty node one level, where we will try to
  solve it instead. Thus, this is the one case where we need recursion
  (or an iterative approach), while all the others solve the problem
  right away.
  This case is shown in Figure~\ref{fig:propagate-up}.

\begin{figure}[htb]
\begin{center}
\psset{unit=0.7cm,levelsep=7ex,arrowsize=0.1 3}
\begin{pspicture}(-3,-5)(3,1)
\pstree{\Tcircle{7}}
       {
        \pstree{\Tcircle{4}}
               {
                 \pstree{\Tp}{\Ttri{$T_1$}}
                 \pstree{\Tp}{\Ttri{$T_2$}}
               }
        \ProblemTree
       }

\pstree{\Tcircle[fillstyle=solid,fillcolor=black]{\phantom{5}}}
      {
        \pstree{\Toval{4 7}}
               {
                 \pstree{\Tp}{\Ttri{$T_1$}}
                 \pstree{\Tp}{\Ttri{$T_2$}}
                 \pstree{\Tp}{\Ttri{$T$}}
              }
       }
\end{pspicture}
\caption{Merging two nodes and recursing. The sibling and parent are
  merged, and the empty node is pushed up one level, where it still
  needs to be dealt with. \label{fig:propagate-up}}
\end{center}
\end{figure}

\item There is one final case. If the empty node is the root now, then
  it can be safely deleted. Its one child (the root of $T$) becomes
  the new root of the tree, which is now one level smaller.
\end{enumerate}

This covers all the cases; each has a few subcases, depending on which
of the children is empty, but they all follow the same pattern.

Notice that these operations carefully maintain all the properties:
the leaves stay at the same level, each node has one or two children
once we are done fixing problems, and the tree is still a 2-3 search
tree. 

\subsection{A detailed illustration}

To illustrate how to apply the rules given above, let's delete a few
keys from our example tree $T$, but with another key 16 added to it.
The resulting tree is shown in Figure~\ref{fig:added-16}.

\begin{figure}[htb]
\begin{center}
\psset{unit=0.7cm,levelsep=7ex,arrowsize=0.1 3}
\begin{pspicture}(0,-4)(0,1)
\pstree{\Toval{6 14}}
        {
	 \pstree{\Tcircle{4}}
                {
		  \Toval{1 3}
		  \Tcircle{5}
                }
	 \pstree{\Toval{8 10}}
                {
		  \Tcircle{7}
		  \Tcircle{9}
		  \Toval{11 13}
                }
	 \pstree{\Tcircle{19}}
                {
		  \Toval{16 17}
		  \Tcircle{23}
                }
        }
\end{pspicture}
\caption{The tree $T$ with the key 16 added.\label{fig:added-16}}
\end{center}
\end{figure}

Now, let's see what happens when we want to remove the key 14.
As we mentioned earlier, removing keys from internal nodes is a
glorious mess, so we swap 14 with its successor, which is 16 (go right
once, then left after that).
The result of swapping 14 (key to delete) with its successor (16) is
shown in Figure~\ref{fig:swap}.

\begin{figure}[htb]
\begin{center}
\psset{unit=0.7cm,levelsep=7ex,arrowsize=0.1 3}
\begin{pspicture}(0,-4)(0,1)
\pstree{\Toval{6 16}}
        {
	 \pstree{\Tcircle{4}}
                {
		  \Toval{1 3}
		  \Tcircle{5}
                }
	 \pstree{\Toval{8 10}}
                {
		  \Tcircle{7}
		  \Tcircle{9}
		  \Toval{11 13}
                }
	 \pstree{\Tcircle{19}}
                {
		  \Toval{14 17}
		  \Tcircle{23}
                }
        }
\end{pspicture}
\caption{After 16 and 14 have been swapped.\label{fig:swap}}
\end{center}
\end{figure}

Now, the key can be safely deleted from the leaf.
Notice that this maintains the search tree property, as the successor
was safe to take the place of the key we want to delete.
The result of deleting 14 now is shown in Figure~\ref{fig:delete-14}.

\begin{figure}[htb]
\begin{center}
\psset{unit=0.7cm,levelsep=7ex,arrowsize=0.1 3}
\begin{pspicture}(0,-4)(0,1)
\pstree{\Toval{6 16}}
        {
	 \pstree{\Tcircle{4}}
                {
		  \Toval{1 3}
		  \Tcircle{5}
                }
	 \pstree{\Toval{8 10}}
                {
		  \Tcircle{7}
		  \Tcircle{9}
		  \Toval{11 13}
                }
	 \pstree{\Tcircle{19}}
                {
		  \Tcircle{17}
		  \Tcircle{23}
                }
        }
\end{pspicture}
\caption{After the deletion of the key 14.\label{fig:delete-14}}
\end{center}
\end{figure}


Deleting the key 14 was pretty easy: after the swap, its leaf node
contained two keys, which made it safe to just delete a key.
Next, let's see what happens if we remove the key 5.
That key is in a node all by itself, so after we remove it, we have an
empty leaf node. This situation is shown in Figure~\ref{fig:delete-5-1}.

\begin{figure}[htb]
\begin{center}
\psset{unit=0.7cm,levelsep=7ex,arrowsize=0.1 3}
\begin{pspicture}(0,-4)(0,1)
\pstree{\Toval{6 16}}
        {
	 \pstree{\Tcircle{4}}
                {
		  \Toval{1 3}
		  \Tcircle[fillstyle=solid,fillcolor=black]{\phantom{5}}
                }
	 \pstree{\Tp}{\Ttri{\phantom{5}}}
         \pstree{\Tp}{\Ttri{\phantom{5}}}
        }
\end{pspicture}
\caption{After the deletion of the key 5. The now-empty node is shown
  in black.\label{fig:delete-5-1}}
\end{center}
\end{figure}

Here, the empty node has a sibling with two keys 1 and 3, so we can
apply the first rule from our list above. 
In this case, we can move the key from the parent down to fill the
node, and move one of the keys from the sibling up to the parent.
The result is the tree in Figure~\ref{fig:delete-5-2}:

\begin{figure}[htb]
\begin{center}
\psset{unit=0.7cm,levelsep=7ex,arrowsize=0.1 3}
\begin{pspicture}(0,-4)(0,1)
\pstree{\Toval{6 16}}
        {
	 \pstree{\Tcircle{3}}
                {
		  \Tcircle{1}
		  \Tcircle{4}
                }
	 \pstree{\Toval{8 10}}
                {
		  \Tcircle{7}
		  \Tcircle{9}
		  \Toval{11 13}
                }
	 \pstree{\Tcircle{19}}
                {
		  \Tcircle{17}
		  \Tcircle{23}
                }
        }
\end{pspicture}
\caption{Borrowing a key from the sibling via the parent.\label{fig:delete-5-2}}
\end{center}
\end{figure}

Next, let's see how borrowing from a sibling works when a node has two
siblings, and the one with two keys is the furthest.
To see this, let's look at what happens if we next remove the key 7.
The intermediate stage will be as shown in Figure~\ref{fig:delete-7-1}.

\begin{figure}[htb]
\begin{center}
\psset{unit=0.7cm,levelsep=7ex,arrowsize=0.1 3}
\begin{pspicture}(0,-4)(0,1)
\pstree{\Toval{6 16}}
        {
	 \pstree{\Tp}{\Ttri{\phantom{5}}}
	 \pstree{\Toval{8 10}}
                {
		  \Tcircle[fillstyle=solid,fillcolor=black]{\phantom{5}}
		  \Tcircle{9}
		  \Toval{11 13}
                }
         \pstree{\Tp}{\Ttri{\phantom{5}}}
        }
\end{pspicture}
\caption{Deleting the key 7. The now-empty node is shown in black.\label{fig:delete-7-1}}
\end{center}
\end{figure}

This time, borrowing a key from the sibling still works, but takes a
few more intermediate steps. We're still in the first rule case, but
it works a little differently.
The key 11 from the sibling gets moved up to the parent node; the key
10 from the parent gets moved down to the other sibling, while the key
9 from the other sibling gets moved up to the parent. Finally, we can
move the key 8 down to the empty node. The result is only locally
rearranged, and shown in Figure~\ref{fig:delete-7-2}.

\begin{figure}[htb]
\begin{center}
\psset{unit=0.7cm,levelsep=7ex,arrowsize=0.1 3}
\begin{pspicture}(0,-4)(0,1)
\pstree{\Toval{6 16}}
        {
	 \pstree{\Tp}{\Ttri{\phantom{5}}}
	 \pstree{\Toval{9 11}}
                {
		  \Tcircle{8}
		  \Tcircle{10}
		  \Tcircle{13}
                }
         \pstree{\Tp}{\Ttri{\phantom{5}}}
        }
\end{pspicture}
\caption{Borrowing from a 2-step removed sibling via the parent. This
  requires some moving up and down of keys, but only between the
  siblings and the parent.\label{fig:delete-7-2}}
\end{center}
\end{figure}

Next, let's see what happens when there's no sibling to borrow from. 
To see this, we delete the key 8 next. 
The resulting tree is shown in Figure~\ref{fig:delete-8-1}.

\begin{figure}[htb]
\begin{center}
\psset{unit=0.7cm,levelsep=7ex,arrowsize=0.1 3}
\begin{pspicture}(0,-4)(0,1)
\pstree{\Toval{6 16}}
        {
	 \pstree{\Tp}{\Ttri{\phantom{5}}}
	 \pstree{\Toval{9 11}}
                {
		  \Tcircle[fillstyle=solid,fillcolor=black]{\phantom{5}}
		  \Tcircle{10}
		  \Tcircle{13}
                }
         \pstree{\Tp}{\Ttri{\phantom{5}}}
        }
\end{pspicture}
\caption{Deleting the key 8. The now-empty node is shown in black.\label{fig:delete-8-1}}
\end{center}
\end{figure}

Now, we cannot borrow from a sibling directly. But here, fortunately,
the empty node has a parent with two keys, so it's possible to apply
the second rule and borrow a key from the parent, and another one from
the immediately neighboring sibling. 
The resulting tree is shown in Figure~\ref{fig:delete-8-2}.

\begin{figure}[htb]
\begin{center}
\psset{unit=0.7cm,levelsep=7ex,arrowsize=0.1 3}
\begin{pspicture}(0,-4)(0,1)
\pstree{\Toval{6 16}}
        {
	 \pstree{\Tp}{\Ttri{\phantom{5}}}
	 \pstree{\Toval{11}}
                {
		  \Toval{9 10}
		  \Tcircle{13}
                }
         \pstree{\Tp}{\Ttri{\phantom{5}}}
        }
\end{pspicture}
\caption{Borrowing a key from the parent.\label{fig:delete-8-2}}
\end{center}
\end{figure}

Next, let's look at the effect of deleting the key 17
from the tree. The first step of deletion is shown in
Figure~\ref{fig:delete-17-1}.
\begin{figure}[htb]
\begin{center}
\psset{unit=0.7cm,levelsep=7ex,arrowsize=0.1 3}
\begin{pspicture}(0,-4)(0,1)
\pstree{\Toval{6 16}}
        {
	 \pstree{\Tcircle{3}}
                {
		  \Tcircle{1}
		  \Tcircle{4}
                }
	 \pstree{\Toval{11}}
                {
		  \Toval{9 10}
		  \Tcircle{13}
                }
	 \pstree{\Tcircle{19}}
                {
		  \Tcircle[fillstyle=solid,fillcolor=black]{\phantom{5}}
		  \Tcircle{23}
                }
        }
\end{pspicture}
\caption{Deleting the key 17. The now-empty node is shown in black.\label{fig:delete-17-1}}
\end{center}
\end{figure}

Now, there is no easy way to fix the problem at this level:
we have to apply the third rule and escalate the problem one level up. 
To do so, we combine the one key in the parent node with the one key
in the sibling node to form a node with two keys, and make the parent
empty. 
In other words, we produce the tree shown in Figure~\ref{fig:delete-17-2}:

\begin{figure}[htb]
\begin{center}
\psset{unit=0.7cm,levelsep=7ex,arrowsize=0.1 3}
\begin{pspicture}(0,-4)(0,1)
\pstree{\Toval{6 16}}
        {
	 \pstree{\Tcircle{3}}
                {
		  \Tcircle{1}
		  \Tcircle{4}
                }
	 \pstree{\Toval{11}}
                {
		  \Toval{9 10}
		  \Tcircle{13}
                }
	 \pstree{\Tcircle[fillstyle=solid,fillcolor=black]{\phantom{5}}}
                {
		  \Toval{19 23}
                }
        }
\end{pspicture}
\caption{Combining a parent key with a sibling key, and passing the
  problem of the empty node up one level.\label{fig:delete-17-2}}
\end{center}
\end{figure}

At the next higher level, we now have a parent with two keys, so we
can apply the second rule and borrow from the parent to fix the
problem, giving us the tree shown in Figure~\ref{fig:delete-17-3}.

\begin{figure}[htb]
\begin{center}
\psset{unit=0.7cm,levelsep=7ex,arrowsize=0.1 3}
\begin{pspicture}(0,-4)(0,1)
\pstree{\Tcircle{6}}
        {
	 \pstree{\Tcircle{3}}
                {
		  \Tcircle{1}
		  \Tcircle{4}
                }
	 \pstree{\Toval{11 16}}
                {
		  \Toval{9 10}
		  \Tcircle{13}
		  \Toval{19 23}
                }
        }
\end{pspicture}
\caption{Borrowing from the parent to resolve the empty node in the
  middle of the tree.\label{fig:delete-17-3}}
\end{center}
\end{figure}

Of course, we were a bit lucky --- if the parent had had only one key,
we may have had to pass the problem up yet another level, possibly all
the way up to the root. Once the root is empty, though, it is safe to
simply delete it (the fourth rule), thereby reducing the height of the
tree by 1.

\subsection{Running Time}
When we look at the running time, we see that rearranging the tree is
only moving a constant number of pointers and keys (and values)
between nodes. Thus, all of the operations above take time $O(1)$.
If we apply rules 1, 2, or 4, we are thus done in $O(1)$. In case 3, we may
continue iteratively or recursively, but we perform a rearrangement
operation at most once per level of the tree. Since we already worked
out that a 2-3 Tree has at most $\log_2(n)$ levels, the running time
is $O(\log n)$ for each of the operations.
Thus, in summary, we get that using a 2-3 tree, all operations take
time $\Theta(\log n)$. (It is not hard to construct trees on which you
do need to propagate up all levels for an insertion or a removal.)

For many data structures, we say that which one to choose depends on
the context. It is generally agreed that search trees (and Hashtables,
which we will get to next) are simply better implementations of
Dictionaries than those based on arrays, vectors, or lists. If you
need an associative structure, don't use a linear structure --- use a
tree or Hashtable.
