Suppose that a friend has coded an integer (or perhaps templated) \code{List}
class based on linked lists and you want to create a deluxe version of it. 
The deluxe class \code{DeluxeLinkedList} would do the same thing
except that it has an \code{isEmpty} function and an upgraded
version of your friend's \code{print} function.
There are two natural ways of accomplishing this. 
\begin{enumerate}
\item You could just change your friend's code everywhere. But your
  friend may not like this; perhaps, he/she needed the particular format
  of printing the list, and you just destroyed his/her code.
  In particular if you are working on a project together,
  both versions may be needed at the same time,
  so it is not just about choosing the ``better'' version.
\item You could just copy and paste your friend's code and write your own
  modifications into the copied version. But that also has problems:

\begin{itemize}
\item There are now two versions of the original code, so changes made
  to one have to be carefully copied over to the other.
  In particular, if your friend discovers a bug in his/her code,
  it now has to be fixed in two places (or more, if you made different versions).
\item Copying/pasting is very inelegant as a general programming
  technique.
\end{itemize}
\end{enumerate}

Both methods also require you to have access to your friend's complete source
code in the first place, and not just the compiled object files and headers.

The clean solution to this problem is called \todef{inheritance},
and it is one of the most central notions of object-oriented design/programming.
Given a class A, inheritance allows you to define a new class B that
``inherits'' (automatically copies) all data and functions from A. 
Further, it allows you to modify and add functions as necessary, and
overwrite existing functions. The basic syntax is as follows: 

\begin{verbatim}
class B : public A
{
  ...
}

\end{verbatim}
This code basically just copy-pastes everything from A into B
(except what you overwrite).
The difference is that as soon as something is changed in A,
the changed version is also used in B,
so you only need to fix things in one place, and
--- unless you use templates (which create some problems) ---
you do not even need access to the implementation of A, just the
header file and compiled version.

When you use inheritance, the class you are inheriting from (here, A) is
called the \todef{base class} or \todef{superclass}, while the class
that is doing the inheriting (here, B) is called the \todef{subclass}.

As far as our thinking is concerned, the code for A is copied into B.
Therefore, the same rules apply for overloading functions as are in
effect when you write your own class from scratch:
you may overload function names so long as the functions have
different signatures (numbers or types of arguments).
But, as we will see, you also get to \emph{replace} one function in A
with another function in B which has the same signature.

Now that you hopefully understand the basics of inheritance, 
let us build \code{DeluxeLinkedList} by inheriting from
\code{IntLinkedList}. 
First, we add a new function to test whether the list is empty:

\begin{verbatim}
bool DeluxeLinkedList::isEmpty () {
  return (this->head == nullptr);
}
\end{verbatim}

Then, we replace the \code{print} function (which had previously used
recursion to print the list in reverse order) with the following:

\begin{verbatim}
void DeluxeLinkedList::print() {
  for (Item *p = head; p != NULL; p = p->next)
     std::cout << p->value << " ";
  std::cout << std::endl;
}
\end{verbatim}

In addition, we put the two function headers inside 
\begin{verbatim}
class DeluxeLinkedList : public LinkedList
{
  public:
    void print ();
    bool isempty ();
}
\end{verbatim}

When we try to compile this, the compiler will note an error,
and claim that our \code{DeluxeLinkedList} does not have a variable
called \code{head}.
But we inherited from \code{LinkedList},
which does have a variable called \code{head}!

The problem is that \code{head} is declared as \code{private} in
\code{LinkedList}. The keyword \code{private} makes the variable
very restricted: not even an inheriting class gets to access the
variable (though of course the data item is stored in the class).
A version that resembles \code{private} but allows all inheriting
classes to access the variable directly is \code{protected}.
Here is a summary of who is and is not allowed to access variables or
functions of a class:

\begin{description}
\item[public:] everyone (other classes, this class) can access the field.
\item[private:] only objects of the same class can access the
  field. One object of class A may access the private fields of
  another object of class A.
\item[protected:] only objects of the same or inheriting classes can
  access the field.
\end{description}

When we replace the word \code{private} with \code{protected} in the
definition of \code{LinkedList}, everything works afterwards.

\section{Member Visibility, Multiple Inheritance, and calling Base Class Methods}

Inheritance raises a few interesting issues,
such as whether subclasses (the inheriting ones) can access methods
and variables from the superclasses,
how to inherit from multiple classes,
and how to explicitly call a member function of the base class when
you are also overwriting it in the inheriting class.
We will discuss those here.

\subsection{Inheritance and Visibility}
You were probably wondering why we were writing 
\code{class B : public A}; in particular, why we needed the word
\code{public} here.
When you have a class B inheriting from A, sometimes, you want to 
restrict access to variables that were \code{public} in A;
that is, you want to \emph{make} them \code{private} or
\code{protected} in B.
For instance, you may  want to inherit the functionality of a linked
list to use it, but show the world a different set of functions to
call, and keep the other ones only for internal use.
Then, you would want them to become private. 
To distinguish more precisely what should become private and what
should remain public, you can inherit in one of three ways:

\begin{description}
\item[public:] All elements in A have the same protection in B as
  they had in A, so everything stays as it was.
\item[protected:] Private elements of A remain private in B, but
  protected and public elements of A both become protected in B. In
  other words, the minimum protection level for elements of A is now
  \code{protected}.
\item[private:]	All elements of A become private in B.
\end{description}

So by writing \code{class B : private A} instead, we would ensure that
whatever B has inherited from A cannot be accessed from any other classes:
neither those inheriting from B, nor those using objects of type B.

Of course, it is important to remember that even when you inherit
privately, all variables and functions that were part of A are still
there in B --- they are just not accessible to the outside world.
But you may have some functions in B that use variables of A that are
now private
As a result, it is still possible to access those now-private
members indirectly, just not directly.

\subsection{Multiple inheritance}

C++ allows a single class to inherit from multiple classes.
We strongly advise you to not use this feature, as quite a few things
can go wrong. For instance, if the two classes you are inheriting from
have variables or functions of the same name, you get clashes.
For some discussion, look up ``C++ Diamond of Dread.''
If you find yourself inheriting from more than one class, it is often
a sign that you should rethink your program's class structure. 

The most frequent ``legitimate'' use of multiple inheritance is when
you want to genuinely inherit the functionality of one class, and add
functions to meet all the specs of a pure abstract class.
(We will talk about pure abstract classes momentarily.)
Then, from the second class, you really do not ``inherit'' per se,
but you just implement the functions that that class wants. 
Unfortunately, in C++, the only way to do that is multiple
inheritance. 

In other more naturally object-oriented languages (such as Java),
purely abstract classes which specify functions you would like to see
in your class are distinguished from actual classes;
in Java, they are called \todef{Interface}.
Then, you can say that your class \todef{inherits} one class,
and \todef{implements} another interface.
Unfortunately, C++ does not have a mechanism for this, so sometimes,
you are stuck with multiple inheritance.

\subsection{Calling Base Class Methods}
We overwrote the \code{print} function in our implementation of the
\code{DeluxeLinkedList} class. What happens if someone actually wanted
to call the \code{LinkedList} version of \code{print}?
For instance, suppose that instead of printing the elements in correct order,
we still wanted to stick with the reverse order of \code{LinkedList},
but we wanted to print a message before and after.
C++ allows you to do that, by explicitly putting the name of the class
in front of a call.
Generally, to call a function belonging to class A, you would write
\begin{verbatim}
  A::functionName(parameters);
\end{verbatim}

So specifically for the \code{print} function implementation, we would
write
\begin{verbatim}
DeluxeLinkedList::print () {
  std::cout << "This is the deluxe version of print\n";
  std::cout << "***********************************\n";
  LinkedList::print ();
  std::cout << "***********************************\n";
}
\end{verbatim}
Notice that the \code{print()} function does not recursively call
itself, as it calls the other version.
Of course, if we did not add a function of the same name
(as we did with \code{print} here),
we could leave out the \code{A::} part,
and just call it as \code{functionName(parameter)},
since the function is inherited, and part of our new class B.

\section{Class Relationships}
When you build a large project, you will often have objects of dozens,
or even hundreds, of different classes interacting in different ways.
While at first this may seem overwhelming, if you define your class
hierarchies and relationships carefully,
it actually helps you organize and understand the way your code works,
compared to just writing thousands of lines of code without much structure.
Forming high-level conceptual ideas about how your project is
organized can be really helpful.

The following three are very standard relationships between classes.%
\footnote{There is much more to learn along these lines:
  a topic called ``design patterns'' explores many different standard
  patterns in which classes interact.
  We will see several simple examples later in this class,
  including Iterators in Chapter~\ref{chap:iterators}.
  In the context of Qt and event-based programming, discussed in
  Chapter~\ref{chap:qt}, you also get interesting interaction patterns
  between classes.}

\begin{description}
\item[Is-A:] We say that B \todef{is a(n)} A if B is a more specific
version of A; B has all the functionality of A and then some more. 
For instance, every cat \emph{is a} mammal: it has all the abilities
of a generic mammal, plus many cat-specific ones.

When you write code, \emph{is a} relationships are typically
implemented by using public inheritance: you want all the
functionality of A to remain available,
while adding more functions or data to B.

\item[As-A:] B \todef{as a(n)} A refers to implementing a class B using
  the functionality of A. B will typically look like something
  completely different to the outside world, but internally, it is
  implemented by using just A's functionality.
For instance, when you sleep on someone's couch,
you are implementing the functionality of ``bed'' using a ``couch''.
(You implement ``bed'' \emph{as a} ``couch''
--- even though that is the opposite as you would refer
to it with the verb ``use''.)
If instead, you sit on the couch, you may be implementing the
functionality of ``chair'' using a ``couch''. 

When you write code, \emph{as a} relationships are typically
implemented by using protected or private inheritance:
you do not want other classes to see the underlying functionality;
instead, all you want to expose to whoever uses your class is the new
parts you are implementing.
In the examples above, that would be ``bed'' or ``chair;''
the fact that a ``couch'' was used internally should/will not be visible.

\item[Has-A] B \todef{has a(n)} A if one of the fields of B is of type A. 
For instance, your car probably \emph{has a} radio.
That does not mean that your car itself has a function ``switch station,''
but that it contains an object which itself has that function.

\emph{Has a} relationships require no inheritance;
they simply involve creating a member variable of type A in B. 
\end{description}

The boundaries between \emph{has a} and \emph{as a} can be pretty
fluid, and sometimes, it makes sense to think about solving a
particular design issue in either way. 
For instance, are we implementing a \code{Map} \emph{as a}
\code{List}, or does a Map \emph{contain} a List?
In implementing it, one can often use these two nearly
interchangably.
This distinction tends to give students a bit of trouble,
so make sure you understand it.

\section{Static vs.~Dynamic Binding}
We saw that when one class \code{B} inherits from another class \code{A},
it can overwrite/replace some of the functions that already existed in
\code{A}.
For instance, our \code{DeluxLinkedList} inherited from the more basic
\code{LinkedList},
and we changed the \code{print} function from one to the other.

One of the nice things that (public) inheritance does for us is that when
\code{B} inherits from \code{A},
then each object of type \code{B} has all the functionality of an
object of type \code{A}.
Therefore, we can assign objects of type \code{B} into variables of
type \code{A}.
In other words, for our example of \code{DeluxeLinkedList}
inheriting from \code{LinkedList},
the following would be valid code:

\begin{figure}[htb]
\begin{verbatim}
    LinkedList *p;
    DeluxeLinkedList *q = new DeluxeLinkedList ();
    p = q;
    p->print();
\end{verbatim}    
\caption{Which version of \code{print} is called? \label{code:InheritingLinkedList}}
\end{figure}

Because by our inheritance, every \code{DeluxeLinkedList} is-a
\code{LinkedList}, we can assign objects of type
\code{DeluxeLinkedList} into variables of type \code{LinkedList}.
And because objects of type \code{LinkedList} have a
\code{print} function, we can call it.

Notice that this assignment only works in this direction:
assigning a more specific object to a less specific variable.
This way, we can be sure that the actual object has all of the
required functionality.
It does not work the other way around.
Think about a real-world example:
we have one class ``human'' and one class ``student,''
which inherits from ``human.''
Every student has all the abilities of a generic human,
plus some extra ones.
When you need a human for a particular task (e.g., ``donate blood''),
you can just substitute a student if you want.
But when you need a student (e.g., for ``take exam''),
you cannot substitute just any human,
since they may not be of the right type to
know how to execute the necessary functions. 

What we just discussed works only for \todef[inheritance]{public} inheritance. 
If a class inherits from another privately or protectedly,
it does not have the same methods and variables visible,
so you may not be able to use it.
For instance, if you were planning to use your ``couch'' for sitting on it,
and you implemented ``bed'' as-a ``couch,'' then the bed is not
helpful:
by hiding the ``couch'' functionality,
it cannot be used as a couch any more.

Now that we know that we can use a \code{DeluxeLinkedList} anywhere
that a \code{LinkedList} has been called for,
the following question arises:
since \code{DeluxeLinkedList} and \code{LinkedList}  both
implement different versions of a \code{print} function,
which one is called in the last line?
We could try out two arguments:
\begin{enumerate}
\item The compiler knows that \code{p} points to an object of type
  \code{LinkedList}. It cannot really figure out by code
  analysis that it so happens that this time (not even always),
  \code{p} points to an object that is actually of type
  \code{DeluxeLinkedList}. 
  So the call must be to the version of \code{print} defined in \code{LinkedList}.
\item While the compiler does not know at compile time what exactly
  \code{p} points to, when it comes time to actually \emph{call} a
  function (during execution), it must know the precise object that is
  pointed to by \code{p},
  and it so happens to be of type \code{DeluxeLinkedList}. 
  Therefore, the call will be to the version of \code{print} defined in
  \code{DeluxeLinkedList}.
\end{enumerate}

It turns out that both are by themselves valid arguments.
The first version is the default.
If you do not put any additional instructions in the code,
C++ will call the version of the function corresponding
to the declared type of the variable you are using.
This is called \todef{static binding}.

But sometimes, you would really like the second option to happen
--- you will see some examples in a moment.
You can tell the compiler to do this,
by declaring the \code{print} function to be \todef[function]{virtual}.
You do this by just adding the word \code{virtual} in the definition of
the function (both in the parent and child class), as follows:
\begin{verbatim}
class LinkedList {
   ...
   virtual void print ();
   ...
}
\end{verbatim}

The keyword \code{virtual} tells the compiler to not try to figure out
at compile time which version of the function to call,
but leave that decision until runtime,
when it knows precisely the type of the object.
This process --- figuring out which version of the function to
call at runtime --- is called \todef{dynamic binding},
as opposed to \todef{static binding},
which is when the compiler figures out at
compile time which version to call.

\section{Pure virtual functions and abstract classes}
Sometimes, you want to declare a function in a class, but not implement
it at all; you will see in a moment why.
Such a function is called a \todef{pure virtual function}.
The declaration lives entirely in the base classes' header files,
and you do not need to care about it at all in the implementation files.
The syntax is 

\begin{verbatim}
class A {
  public:
  virtual void print() = 0;
}
\end{verbatim}

The \code{= 0} part tells the compiler that the class does not
implement this function.
Of course, if a class has one or more pure virtual functions in it,
the class cannot be instantiated.
Such a class is called an \todef{abstract class}.
No object can be created from an abstract class.
After all, if we could create objects,
then what would code like the following do?

\begin{verbatim}
A obj;
obj.print();
\end{verbatim}
What code should be executed when the execution gets to \code{obj->print()}?
There is no function to call.
But if you can never instantiate an abstract class,
why would there ever be any reason to even define one?

\subsection{Why and how to use pure virtual functions}
The main reason to create pure virtual functions is to force
inheriting classes to implement a function.
For instance, in the example above, if we later implement classes
\code{B} and \code{C} which inherit from \code{A},
we have forced both of them to provide a \code{print()} function.
So we can safely call the \code{print()} function on them.

Pure virtual functions and abstract classes are very useful in
structuring one's classes,
and in avoiding code (and error) duplication.
For a standard example, imagine that you are developing a graphics
program that lets you draw basic shapes.
Since all graphics objects (rectangles, circles, polygons, etc.)
share some properties (such as perhaps color, line width, line style,
and a few others)
and maybe also some functionality,
you may want to define a general class
\code{GraphicsObject} which captures those.
You also want each of these objects to provide some functions, such as
perhaps \code{draw} (drawing itself), \code{resize}, or others.
But absent a specific definition of an object type (e.g., circle), you
cannot implement these functions yet.
Yet, you want to force all \code{GraphicsObject} objects you later
define to provide these functions,
so that you can put them all in one big array and process them the
same way.
In other words, you would like to do something like the following:
\begin{verbatim}
  List<GraphicsObject> allMyObjects;
  // code that loads a bunch of different objects into the list.
  for (int i = 0; i < allMyObject.size(); i ++)
     allMyObjects[i].draw();
\end{verbatim}
Even though \code{allMyObjects} may contain many different types of
graphics objects, so long as all of them inherit from
\code{GraphicsObject}, and thus have to implement a \code{draw()}
function, the code shown above will work perfectly fine.

The way to achieve this is to declare pure virtual functions
\code{draw()}, \code{resize()}, and whatever other functions you want.
Now, all classes you define later (like \code{Circle},
\code{Rectangle}, \code{Polygon}, etc.) that inherit from 
\code{GraphicsObject} must implement this function in order to have
objects generated from them.

Abstract classes are also very useful to specify an abstract data
type, and separate its functionality from its implementation.
After all, we have said before that an abstract data type is
characterized by the functions it supports, and there may be many ways
to implement them.
For example, a while back, we specified the ADT \code{Set},
which looks as follows when written as an abstract class.
(Say we are interested in sets of integers.)

\begin{verbatim}
class Set {
  public :
     virtual void add (int item) = 0;
     virtual void remove (int item) = 0;
     virtual bool contains (int item) const = 0;
}
\end{verbatim}

The preceding piece of code only says that any class that wants to
implement a \code{Set} must provide these functions.
Another programmer can then create, say, a \code{LinkedListSet}, an
\code{ArraySet}, a \code{VectorSet}, etc.
All of these classes would inherit from \code{Set} and implement the
functions, though they could do so in very different ways.
By inheriting from the abstract class \code{Set}, these other classes
have made sure to be very easily interchanged, as follows:

\begin{verbatim}
   Set * p;
   p = new LinkedListSet();
   ... // some time later
   delete p;
   p = new ArraySet();
\end{verbatim}
Using dynamic binding here (indicated by the \code{virtual} keyword),
the code has no problem calling the right version of each function,
and we can use different implementations interchangeably.

What would happen if we tried to define something
like a non-virtual unimplemented function?
\begin{verbatim}
  void add (int n) = 0;
\end{verbatim}
This results in an error. There is no point in having a non-virtual
non-defined function. If it is not defined, it cannot be called.
And if it is not virtual, then when it is overwritten later, the compiler
cannot figure out to call the overwriting version, unless the object is
already of type \code{B}, in which case there is no need for inheriting.

\medskip

Another classic reason for using pure virtual functions is that we can 
often avoid rewriting very similar code.
For instance, imagine that we are implementing an abstract data type
(say, \code{List}),
and focus on variants of the \code{insert} function. 
In addition to the basic \code{insert} version, you want to provide
several special functions, like \code{pushback} (which inserts at the
end), \code{pushfront} (which inserts at the beginning), and maye also
others.

Since the additional functions like \code{pushback} can be described
in terms of \code{insert}, you could write code like the following: 

\begin{verbatim}
class IncompleteList {
    public:
       void pushback (int item);
       void pushfront (int item);
       void insert (int n, int item) = 0;
}
\end{verbatim}

You could then implement the \code{pushback} and \code{pushfront}
functions once and for all, and reuse the code in this way. 
Something like the following:

\begin{verbatim}
IncompleteList::pushback (int item) {
  insert (size(), item);
}

IncompleteList::pushfront (int item) {
  insert (0, item);
}
\end{verbatim}
Then, our \code{LinkedList} and \code{ArrayList} and
\code{VectorList} would not have to implement each of the functions
\code{pushback}, \code{pushfront}, from scratch, but would just
need to provide the missing piece \code{insert}.
Because the \code{insert} function is virtual,
whenever in the course of executing \code{pushfront},
the program runs into \code{insert},
it determines at runtime which version of \code{insert} to call. 

This concept --- determining which version of a class member function
to call at runtime --- is called \todef{polymorphism}, which literally
means ``many forms:'' the object stored in a variable could be of one of
many forms, and the execution will do ``the right thing'' for the
current object.

\section{Constructors and Destructors with Inheritance}

Suppose that you have a class B which is inheriting from A. What
happens when you write something like the following piece of code?

\begin{verbatim}
B *myB = new B ();
\end{verbatim}

First, appropriate space is reserved on the heap for the object of
type B. But then, a constructor needs to be called. You will clearly
want to call a constructor for B; in the given code, it will be the
default constructor.
But if A has some private variables (which B inherits),
how will they get initialized?
The answer is that \emph{before} the constructor of the class B is
called, C++ calls a constructor for the class A.
If you do not explicitly specify which constructor for A to call,
C++ will substitute the default constructor.
If you want to call another constructor, you have to say so explicitly.

Here is how you do this.
Suppose that A has a constructor that passes a \code{string} as a parameter.
For B, you want to implement a constructor that gets a \code{string}
and an \code{int}, and uses A's constructor with a string.
You would write code like the following:

\begin{verbatim}
class B : public A {
  public:
    B (string s, int n) : A (s)
     { rest of the code here }
}
\end{verbatim}

% edited up to here

After the name of the constructor, you put a colon and then the
constructor you want to call from the superclass.
Again: it is important to remember that C++ always calls constructors
for all superclasses before the constructor for the class itself.

How about destructors? The same argument suggests that in addition to
the destructor for B, we will also need to call the destructor for A
at some point. After all, A may have dynamically allocated memory for
a private variable, and the only way to safely deallocate it is to
have A's destructor take care of it.

Calls to destructors happen in the reverse order of constructors. That
is, first the destructor of B is called, and then the destructor of
A. This makes sense, since B's stuff is built ``on top of'' A: so we
need to cleanly remove B's stuff first before it is safe to remove the
stuff from A.

For destructors, since classes only have one (with no parameters), you
don't have to worry about explicitly calling one or the other. C++
takes care of the destructor call chain for you; all you need to do is
be aware of it, and not destroy in the destructor of B any data that
you will then again attempt to destroy in A. 

However, something that you should \emph{always} do whenever you have
classes that are inheriting or being inherited from is making the
destructors \code{virtual}. This is very important for the following
reason. Look back at our example in
Figure~\ref{code:InheritingLinkedList}. When we call
\code{delete p}, which destructor is supposed to be called?
Clearly, since we have an object of type \code{DeluxeLinkedList}, it
may have extra data, so it would be important to call its
destructor. But if the destructor is not virtual, then C++ can only go
by the ``official'' type of \code{p}, which is \code{LinkedList}. It
would call the wrong destructor, which may result in a memory leak or
worse. For that reason, any class that has any chance of being
inherited from, and any class that inherits from another, should make
the destructor virtual.

As a general rule for both constructors and destructors, the
constructor and destructor of inheriting classes should take care of
the additional issues for their own class, but leave the other parts
to the superclass.

Also notice that if you have an abstract class, you do not need to
implement any constructors or destructors for it --- in particular,
for a pure abstract class, it makes no sense to build constructors or
destructors.
