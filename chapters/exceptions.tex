\section{Handling errors}
In previous lectures, we implemented a linked list.
Suppose that we had added a function 
\code{int get (int position)} which returns the element at the given
position. 
Now someone else who got hold of our code writes the following:

\begin{verbatim}
    LinkedList *LL = new LinkedList;
    for (int i = 0; i < 10; i ++) LL->append (i);
    cout << s->get(15) << endl;
\end{verbatim}

When \texttt{get(15)} is called, the List does not have 15 elements
yet. What should be returned?
If we ran a loop of 15 steps through the list, the result here will
likely be an attempt at dereferencing a \code{NULL} pointer,
triggering a segfault.
The program will quit, and the person using our Linked List will not
really know what went wrong (except after some debugging). 
Also, simply ending the program may not be the right thing to do here.
What's the right way to deal with this problem?

\begin{itemize}
\item The simplest way is to claim that this is not your problem, and hold
the user responsible for correct usage of your code.
This falls into the broader paradigm ``garbage in, garbage out''. 
This is not a completely unreasonable position to hold; 
the person wanting to use your class should make sure to test whether
the stack is empty before accessing it.
But as we all know, ``should'' does not mean ``will'', and your goal
when writing code is not necessarily to help others develop good
habits, but to develop good code yourself.

\item A second idea would be to return a `magic' value like -1.
This might work when the list is known to only contain positive
values; the programmer could then use the unexpected return to debug
their code. 
However, if the list can contain any integers, there is no ``safe''
magic value. And for a more general list (of a templated type), this
solution is untenable, as we wouldn't even know what to return. 
Further, the programmer might simply choose to ignore bad
returns, which would cause errors to trickle up.

\item A third, and somewhat better, approach is to use the \code{assert}
statement.
In general, \code{assert} statements are good debugging technique.
For instance, we could insert the statement
\code{assert (position < this->size())} as a first line of
\code{get()}. 
(To use \code{assert} statements, you need to \code{\#include<cassert>}.)

An \code{assert} statement simply causes the program to crash when the
asserted condition fails to hold. While this is still a crash, at
least, it gives more useful information, namely, printing in which
line the crash happened, and what condition failed. This helps
significantly in debugging.
However, the program still crashes, which means that it does not have
the option of dealing with the error themselves.
\end{itemize}

These solutions all still involve crashes, which may not be the best
way to handle mistakes, e.g., if you're running the backend database
for a major website, or the control software for a rocket or a nuclear
power plant.
Instead, it would be better to actually signal to the caller of our
function that something went wrong (and what), and give it a chance to
handle the mistake.

\section{Setting Flags}
The way errors are traditionally signaled in C is by making the
function return a \code{bool} instead of the actual value we're
interested in, and passing the actual value by reference in another
variable. The following example illustrates this:

\begin{verbatim}
bool LinkedList::get (int position, int & val) 
{
   if (position < this->size()) {
      Item *p = head;
      for (int i = 0; i < position; i ++)
        p = p->next;
      val = p->value;
      return true;
    } else return false;
}
\end{verbatim}

This is a significant improvement. A slight downside is that the
calling function may still ignore the \code{bool} value, and simply
use the undefined value in \code{val}. 

\section{Exceptions}
The C++ solution is to use exceptions. 
Exceptions allow us to provide feedback about the success of an
operation, in such a way that a caller \emph{must} deal with the
failure condition explicitly.
The basic use of an exception is as follows: 

\begin{enumerate}
\item Upon failure, the method notifies the calling code by
  ``throwing'' an exception.
\item This exception propagates ``up'' through the program stack until
  it reaches a piece of code designed to handle it; it continues
  there. 
\item If no such code is found, the program terminates. 
\end{enumerate}
	
As an illustration, a revised version of our \code{get()} function may
look as follows:
\begin{verbatim}
int LinkedList::get(int position) {
    if (position >= this->size()) throw logic_error ("position was too large!");
    else {
      Item *p = head;
      for (int i = 0; i < position; i ++)
        p = p->next;
      return p->value;
    }
}
\end{verbatim}

Whenever you want to write code that throws/handles exception, you
should \code{\#include<exception>}.
In principle, any kind of object (even an integer, or an
\code{Item}) can be ``thrown'' and treated as an exception.
However, typically, you would like to only use objects that inherit
from the class \code{exception}, since they contain some standard
methods that tell you about what caused the exception.
C++ also provides a few standard ``types'' of exceptions
pre-implemented in \code{\#include<stdexcept>}; for instance, the
class \code{logic\_error} we used above is included in that header
file. Throwing a type of exception whose name (and content) reflect
the type of error that happened helps in understanding what error
occurred.

At the point at which an exception can be reasonably handled, we use
the \code{try}-\code{catch} construct to deal with it.
\code{try} is used for a block that may throw an exception;
\code{catch} provides code to execute in case an exception actually
does happen. If a piece of code would not know what to do with an
exception, it should \emph{not} catch it, and instead let it percolate
up to a higher level, where it may be reasonably handled.
In our earlier list example, the \code{main} function may now look as
follows:

\begin{verbatim}
//main
try {
       cout << s->get(15) << endl;
       cout << "Printed successfully!" << endl;
    } catch (logic_error &e) {
       cout << "A logic error occurred!" << endl;
       cout << e.what();
    }
\end{verbatim}

Here, if/when an exception occurs, the execution of the \code{try}
block terminates (so we never get to the second line about printing
successfully); instead, the program jumps to the first \code{catch}
block that matches the exception. In our case, because a
\code{logic\_error} is thrown, the \code{catch} block matches, so the
program prints the message. It then also prints the string we had
passed into the constructor (i.e., ``head pointer was null!''), since
that's what it returned by the \code{what()} function in the class
\code{exception}. The existence of the \code{what()} function is one
of the reasons to use classes derived from \code{exception}.
If you write your own exception class, you should inherit from
\code{exception} and overload the virtual \code{what()} function to
report what your exception is about.\footnote{We will learn about
  inheritance and virtual functions soon.}

Notice in the above example that \code{e} is an exception object,
which is passed by reference. It has some data in it. 
For example, we just put a message string in it, but we could also
have put in the description of the variables that caused the
exception. 
We could write our own exception object to throw as well; 
for instance, we may decide that we want a specific
exception type to signal if the index is out of bounds.

\begin{verbatim}
class OutOfBoundsException : public exception {
    public:
       int pos;
       OutOfBoundsException (int d) {
              pos = d;
       }   
}
\end{verbatim}

Then, we would throw specifically an \code{OutOfBoundsException} rather
than a \code{logic\_error}.

A \code{try} block can have multiple \code{catch} blocks associated
with it, to deal with different types of exceptions. 
In this case, the program will execute the \emph{first} \code{catch}
block that matches the exception that is thrown.
This means that the more specific types of exceptions should precede
the more general ones.

\begin{verbatim}
LinkedList *L = new LinkedList ();

try {
      cout << L->get(3);
    }
catch (OutOfBoundsException &e)
      { cout << "Array Index was out of Bounds" << endl; 
        // specific treatment when index is out of bounds
      }
catch (exception &e)
      { cout << "General Type of exception" << endl; }
\end{verbatim}

Here, the first block may have some treatment for specifically the
case when the array index was out of bounds, while the second block
may have a less specific treatment (such as just an error message)
when some other, less specific, error occurred.
If we had put the two \code{catch} blocks in the opposite order, the
special treatment for \code{OutOfBoundsException} would never kick in,
as all those exceptions would already be caught in the more general block.
