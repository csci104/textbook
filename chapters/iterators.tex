When we store a set of items in a data structure, one of the things
that we often want to do is be able to output all of them.
For instance, in the case of a set, while our main focus was on adding
and removing items (and test membership), we want to be able to
output all items as well. Ideally, we would also like the process to
be fast. How should we do that?

\section{Some first attempts}
One solution would be to give the programmer (who wants to output
things) access to the internals of our data structure. 
For instance, if it is known that we used a linked list, and we return
the head of the linked list, then starting from the head, it is
possible to go through all items of the linked list.
But of course, this is a terrible solution: it violates everything we
are trying to do by hiding the implementation; the user of our data
structure should only interact with specified functions, and not know
what goes on internally. His/her code would not work any more if we
change the internal implementation to an array-based one.

At the other extreme would be to simply include in our implementation
of the data structure another function \code{print()}, which prints
all items to \code{cout}. But what if we want to print them to a file
instead? Well, we could print them into a string, which could then be
written to either \code{cout} or a file. 
But now what if someone would like them output in a different format?
Or someone only wants to output all the entries that satisfy certain
criteria? We might end up writing a huge number of different output
functions. And then additional functions that might actually process
the items. Clearly, that solution does not work, either.

Another solution would be to essentially provide the interface
of the \code{List} class on top of any data structure: 
allow access to all elements by index, via a function \code{get(int i)}. 
This works better, but it is not very efficient: if the underlying
implementation is via a linked list, then this takes time $\Theta(i)$
for element number $i$, for a total of $\Theta(\sum_{i=1}^n i) = \Theta(n^2)$.
Whereas if we had just traversed the linked list, it would have been 
just $\Theta(n)$. It's also overkill: we don't need to be able to
access elements by index; we just want to be able to go through all of
them.

After thinking about this for a while, the solution one hopefully
arrives at is to have some kind of a ``marker''
that gets pushed through the data, and remembers where it was before.
If the elements of the data structure were in some particular
meaningful order (such as for a \code{List} data structure), then our
marker would traverse them in that order. 
If the order of the elements was not meaningful (for instance, for a
\code{Set} data structure), then they could be output in any order.

The ``marker'' may be implemented differently, depending on how the
data structure itself is implemented. If the data are stored in an
array, then the marker may just be an integer, the index in the array
that we're currently at.
If the data are stored in a linked list, then the marker may be a
pointer to the current element.
Ideally, the implementation of the marker should hide these kinds of
detail, and just expose an interface whereby another function can
retrieve the actual stored elements one by one, without ever having to
know about the marker.

The next question we asked ourselves was where the marker should be
stored. The first ``obvious'' choice was that it may be stored inside
the data structure itself. (That's what we did on Homework 4.)
So things may look something like the
following for our new modified \code{Set} class:

\begin{verbatim}
class Set<T> {
  ... // prior Bag functionality.
  T start(); // starts the traversal, returns first element
             // could be T or T& or T*, depending on how we want to
             // implement it.
  T next(); // advances the marker internally, returns next element
}

Set<T> set; // contains some element already, somehow

for(T elt = set.start(); end not reached; elt = set.next())
   { // process elt, e.g., by printing it; 
   }
\end{verbatim}

We'd have to add a way to test if the end is reached, which is not
very difficult (e.g., return a \code{NULL} pointer or throw an
exception). 
The first problem with the implementation so far is that we might want
to have multiple markers in use at once. 
Then, we couldn't just use \code{start()} and \code{next()}, but would
have to somehow index them.
So the data structure would have to internally store a set of markers
(e.g., as a set/map or something). 
It's doable, but gets a bit cumbersome.

The other issue is that because the state is stored internally in the
\code{Set}, the \code{first()} and \code{next()} functions couldn't be
\code{const}. Thus, we couldn't print any data structure that was
passed as \code{const}, even though in actuality, printing doesn't
change the contents of a data structure.

For both reasons, the more ``standard'' solution is to define a
separate class, whose objects have the ability of traversing 
the elements of the data structure. 

\section{Iterators}
Such a class is traditionally called an \todef{iterator}, so we will
use that term instead of ``marker'' from now on. 
Since an iterator is a class/object encompassing one marker, it must
provide functionality to advance, to dereference (i.e., read the
data), and compare for equality or inequality, in order to terminate a
loop.
For readability (by analogy with traditional \code{for} loops), these
functions are usually implemented with operator overloading.

The class which is being iterated over must have the following
functionality:
\begin{itemize}
\item Return an iterator initialized to the beginning.
  This method is usually called \code{begin()}, and returns an iterator.
\item Return an iterator initialized to the end, so that we can test
  when we are done. This method is usually called \code{end()}, and
  returns an iterator.
\end{itemize}

Suppose that for our \code{Set<T>} class, implemented, say, using a
linked list internally, we want to define an iterator
\code{Set<T>::Iteratator} (we'll see momentarily how to do that).
Then, in the \code{main()} function, our use of an iterator may look
as follows:

\begin{verbatim}
int main()
{
    Set<int> LLS;
    for (int i=2; i<10; i++)
        LLS.add(i*i);
    LLS.remove(49);

    for (Set<int>::Iterator si = LLS.begin(); si != LLB.end(); ++si)
    { 
        cout << *si << " ";
    }
    return 0; 
}
\end{verbatim}

Here, as you can see, we use comparison to the \code{end()} iterator
to test whether we have passed the last element. 
To advance the iterator \code{si}, we use operator overloading, so
that it looks as much as possible like a typical loop in which we would
increment an integer. Similarly, we use \code{*si} to dereference the
current value of the iterator, just like we do with pointers.
So we're trying to make the code look as much as possible like the
following:

\begin{verbatim}
   int a[100];
   for (int *x = a; x < a + 100; ++x)
      cout << (*x) << " ";
\end{verbatim}

One important thing to keep in mind about iterators is that in
general, we want to avoid modifying the underlying data structure
while iterating over it. Most data structures do not guarantee what
will happen if you try that, and it's always a bad idea to execute
operations whose results are unpredictable.
For instance, some implementations of an iterator may have copied all
of the data from the data structure over (so they won't be affected by
any change), while others may look at the data structure in real time
(and will be affected).

\section{Implementing an Iterator}
Let us first see how we would implement an \code{Iterator} class for a
\code{Set<T>}.
First, we want to think about what data need to be stored internally.
We need to have the marker; and since we are implementing the
\code{Set<T>} using a linked list, that marker will be a pointer to an
\code{Item<T>}.
But we also need another field: we need to remember which data
structure the marker is traversing. For instance, two markers both of
which point to element 7 may not be the same if they are pointing to
element 7 of \emph{different arrays} (for implementations using
arrays instead of linked lists). So we also need to store a
pointer to the data structure being traversed.
The signature thus has two private fields for data, and a few methods
discussed below:
\begin{verbatim}
template<class T>

class Set { 
  public:

  class Iterator {
    private:
      const Set<T> *whoIBelongTo;
      Item<T> *current;

      Iterator(const Set<T> *s, Item<T> *p)
       {
          whoIBelongTo = s; current = p;
       }

    public: // functions listed below
       friend class Set<T>;

       T & operator* () const
          {  return current->value; }

       bool operator== (const Set<T>::Iterator &other) const
       {    return (current == other.current && whoIBelongTo == other.whoIBelongTo); }

       bool operator!= (const Set<T>::Iterator &other) const
       {    return (!this->operator== (other)); }

       Set<T>::Iterator operator++ () const
       { 
           current = current->next;
           return *this; 
       }
   }
}
\end{verbatim}

A few notes are in order about the implementation:
\begin{itemize}
\item Notice that we implemented the class as a nested class inside
  \code{Set<T>}, which means it will be referenced by
  \code{Set<T>::Iterator}. There's really nothing difficult about
  nested classes, and they make sense particularly for things such as
  iterators, where one class only is useful for another.
\item Why did we make the constructor \code{private}? The reason is
  that we don't want just anybody to generate iterators for a given
  set and pass weird pointers. The only class to be allowed to
  generate a new iterator should be the set itself. By making the
  \code{Set<T>} a \code{friend}, we've made it possible for the
  \code{begin()} and \code{end()} functions of \code{Set<T>} to call
  the constructor.
\item For the \code{==} operator, we compare the pointers
  (\code{current} vs.~\code{other.current}), rather than their data
  (\code{current->data} vs.~\code{other.current->data}).
  This makes sense because we might have a list that contains the same
  number multiple times; just because it does does not make a marker
  on the 5th element equal to a marker on the 8th.
  Also notice that we are comparing to make sure that both iterators
  are traversing the same data structure.
\item For the \code{!=} operator, we took the easy way out. 
  Instead, we could have implemented it from scratch (using DeMorgan's Theorem).
\item For the increment operator, we not only want to move the marker,
  but also have to declare it of type \code{Set<T>::Iterator} (rather than,
  say \code{void}). The reason is that people may want to write
  something like \code{cout << *(++bi)}, where the iterator is moved
  and immediately dereferenced. While some people
  think that this is not good coding style, you should still support
  it.
\end{itemize}

Now that we have implemented the \code{Set<T>::Iterator} class, we can
augment our \code{Set<T>} with two functions to return a start and
end iterator:

\begin{verbatim}
Set<T>::Iterator Set<T>::begin()
{    return Set<T>::Iterator(this, head); }

Set<T>::Iterator Set<T>::end()
{    return Set<T>::Iterator(this, NULL); }
\end{verbatim}

Again, a few notes on what we did here:
\begin{itemize}
\item Remember that the C++ keyword \code{this} is determined at
  runtime and always contains a pointer to the object from which it is
  called. In other words, an object can use the word \code{this} to
  find out where in memory it resides.
  
  In our case, \code{this} is used to tell the iterator the identity
  of the object it is traversing. The other argument is just the head,
  i.e., where the traversal should start.
\item For the \code{end()} function, we give the \code{NULL} pointer
  as the \code{current} element of the iterator.
  The reason is that the \code{end()} iterator should encode the case
  when the \code{for} loop has reached the end of the linked list,
  which happens when the pointer is a \code{NULL} pointer. 
  If instead, we wrote \code{return Set<T>::Iterator(this, tail);},
  then by looking at what happens in our \code{main()} function above,
  you'll notice that the last element of the set wouldn't be processed.
\end{itemize}

Again, notice how iterators let us go through all elements of a
container without knowing any internal details about it. 
Only the \code{Set<T>} and the \code{Set<T>::Iterator} need to know
how \code{Set} is actually implemented.

The implementation of an iterator for a \code{Set<T>} based on a linked
list was pretty straightforward. If we think about how we would
implement an iterator for a \code{Set<T>} built using an internal array,
we would quickly discover that the iterator will store internally a
pointer to the set, as well as an integer to be used as an array
index. 
Let's say that internally, our \code{ArraySet<T>} stores its data in a
private array \code{T *a}.
Now, the dereferencing would have to be of the form
\begin{verbatim}
T operator* () const
{    return whoIBelongTo->a[current]; }
\end{verbatim}

This means that at the moment we want to return an item, we need
access to a private variable in the data structure.
Typically, we need to use \code{friend} access to ensure this, but
when a class is defined within another (such as here
\code{Set<T>::Iterator} inside \code{Set<T>}), it automatically is
considered part of the parent class, and has full access to all the
private and protected fields.

\smallskip

In our implementation, we only wrote a pre-increment operator. You may
wonder how one would overload the post-increment one? After all, by
writing 
\begin{verbatim}
Iterator operator++ () { ... }
\end{verbatim}
we'd be overloading the pre-increment one. How does C++ tell them
apart? The answer is basically a hack. To implement the post-increment
operator, you write
\begin{verbatim}
Iterator operator++ (int dummy) { ... }
\end{verbatim}
\code{dummy} is just a placeholder variable. You don't actually pass
any integers into your \code{operator++}. It's just a way you tell C++
\emph{which} operator you are overloading.

Also, notice that if \code{T} were a struct, we may want to access its
fields, so we may end up writing \code{(*it).field1} when
dereferencing the iterator. Typically, we'd like to write instead
\code{it->field1}. You can in fact do that if you overload the
\code{->} iterator. What we gave here is the minimum functionality you
should implement for your iterator, but if you at all think that
you'll use it again later, or that other people may use it, you should
implement a few more operators to make it look as much as possible as
an actual pointer as you can.

\section{The Bigger Picture}
\subsection{Design Patterns}
So far in this class, we had seen pretty standard ways for objects of
different types to interact: inheritance, one class containing fields
of another class's type, one class throwing a member of an
\code{exception} class and that's about it.
This is our first example of classes interacting in more complex ways:
we have one class whose ``job'' it is to traverse another class and
provide some functionality (access to elements) that is not naturally
handled within the original class. In other words, we have abstracted
one particular functionality out of one class into another.

Once we absorb the ideas of object-oriented programming more fully, it
becomes really fun to think about objects as each having their own
identity, filling a role in a larger system of components, and
interacting with each other in particular ways.
Since the idea of an Iterator is fairly useful generally, it has been
identified as one of many \todef{Design Patterns}.
Design patterns are an important way of identifying frequently
occurring ways in which objects interact, and they help in designing
well-structured larger software systems that can be extended easily
later. As such, they are an important part of the field of
\todef{Software Engineering}.
We will discuss them in a bit more detail in
Chapter~\ref{chap:design-patterns}. 

\subsection{Foreach, Map, Functional Programming, and Parallelization}
We have seen that for essentially all data structures (e.g., ones in
STL), we will now want to use iterators and write loops of the
following form:

\begin{verbatim}
for (Iterator it = structure.begin(); it != structure.end(); ++it)
   { 
      // process *it
   }
\end{verbatim}

Since the code will always have the exact same form, except for
different ways of processing the elements, we could try to shortcut it
even more. By implementing the processing of \code{*it} into a
function \code{f}, we could implement a function:
\begin{verbatim}
void for_each (Iterator start, Iterator end, Function f)
{
  for (Iterator current = start; current != end; ++current)
     f(*current);
}
\end{verbatim}

In fact, C++ implements an almost identical function that has only a
few small differences. To use C++'s function, your iterator has to
inherit from the abstract class \code{iterator} (for which you 
\code{\#include<iterator>}). You'll have to implement your function,
and pass it in. You have already seen this basic idea of passing a
function pointer in an earlier homework --- this is the first time we
see it as a more general idea.

While using the \code{for\_each} function would only save us a little bit of
typing, it has a conceptual advantage: it clarifies what is going
on in your code. For instance, by using this function, you are making
it clear that you are simply applying the same function to each
element, and there is no state transferred from one iteration to the
next. In fact, this (and a few similar) types of structure appears so
often that it comes pre-implemented like this in STL.

This kind of construction, where you pass a function as a parameter to
another function and then apply it inside, is one of the key
constructs of \todef{functional programming}. 
The particular case in which one function is applied to each element
of a collection is called the \todef{map} paradigm. In other words,
\code{for\_each} implements the map paradigm in C++.

Thinking a little further, the map paradigm is half of the
\todef{map-reduce} paradigm of parallel processing. This is one of the
really popular techniques for massively parallelizing Big Data
computations over large numbers (hundreds of thousands) of machines,
at the heart of what Google and other companies do.
The idea is to farm out a lot of parallel computation on individual
elements to machines (via map), and then combine the results into the
final output (called \todef{reduce}).
If you use constructs such as \code{for\_each}, then you make it much
easier for your code to be transferred into map-reduce code, and
thus parallelized later. 
Notice that this is where it comes in handy that the different
iterations of the loop do not interact with each other, and we instead
have each element processed by itself; by doing this, we ensure that
the different parallel machines do not need to communicate with each
other. (Network usage is slow.)

In addition to the \code{for\_each} loop, if you use iterators, you
also have some other standard functionality available: you can add up
all elements, find the maximum, and a few others, simply by passing
iterators into some pre-programmed functions. Check the textbook for
some examples, in the context of the discussions of iterators and STL.
