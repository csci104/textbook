In the previous chapter, we learned about Abstract Data Types.
The notion of an Abstract Data Type goes hand in hand with
Object-Oriented Design.
In Object-Oriented Design of programs, we group together data items
with the operations that work on them into \todef{classes};
classes are data types, and we can then generate instances of these
classes called \todef{objects}.
The objects have both the data inside them and the
functions that operate on the data.
When specifying a class, we must take good care to keep two things
separate:
\begin{enumerate}
\item the \todef{specification} of the functions that other
classes can call. This is like the specification of an abstract data
type, and usually given in a header (\code{.h}) file.
\item the \todef{implementation} of how the functions are actually doing
their job, and how data items are stored internally.
This is in a \code{.cpp} file, and hidden from other classes.
Hiding it allows us to change the implementation without needing to
change the implementation of other pieces of code that use the class.
\end{enumerate}
We will see much more about classes and object-oriented design in the
next few chapters.

A \todef{class} maps almost exactly to an Abstract Data Type;
you can think of it as essentially a \code{struct} with functions in it.
An \todef{object} is one data item whose type matches a class;
think of a class as the abstract notion of a car and the operations it
supports (accelerate, slow down, honk the horn, switch on/off the
lights, $\ldots$),
while an object is a particular car that implements these properties.
So while there is only one ``car'' class,
there can be many objects of type ``car''.

Remember how we implemented Linked Lists in Chapter~\ref{chap:linked-lists}? 
Inside our \code{main} function, we had a variable for the head of the
list, and we provided functions for appending and removing items from
the list, as well as for printing (or processing) items.
In a sense, the data storage and the functions belong together. 
The \code{append} or \code{remove} function is useless without the
data to operate on, and the data do not help us without those
functions to operate on them.
Classes are exactly the way to group them together,
and get rid of the somewhat awkward global variables for the head (and tail).

Once you get more experienced with object-oriented design ideas,
you will start thinking of your objects not just as pieces of code,
but real entities which sometimes interact with each other in complex ways,
notifying each other of things going on, helping each other out, and so on.
One sign of a more experienced programmer in an
object-oriented language (such as C++ or Java) is to be able to think
of code not just as many lines in a row, but as almost an eco-system
of different objects, each fulfilling its own little well-defined
role, and interacting with each other in interesting and productive ways.%
\footnote{For those a little more interested in programming language
  history, it should be noted that in a sense, we are presenting you
  with a mix of two different motivations for object-oriented design.
  One school (leading the development of the language SmallTalk)
  advocated object-oriented design as the ``natural'' way to implement
  data structures, as we did at teh beginning of the chpater.
  The other school (leading the development of the language Simula)
  felt that designing code was more natural if the code was structured
  to match entities in the real world. For instance, when designing a
  computer system for banking, one would have classes for customers,
  ATMs, checking accounts, etc. Both are very fruitful views and good
  reasons to design object-oriented code.}

\section{Header Files and Declarations}

As a first cut, our class declaration for a linked list of integers
may look as follows:

\begin{verbatim}
class IntLinkedList {
  void append (int n);
  void remove (Item *toRemove);
  void print ();

  Item *head;
}
\end{verbatim}

Notice that the class includes both functions and data (\code{head}).
Because the class contains the pointer \code{head} itself,
we do not need to pass the head of the list to the function any more,
as we did earlier.

Of course, we still need to implement the functions.
But before doing so, we notice that this is what the rest of the
world really needs to know about our class: they need to know the
functions to invoke and (maybe) about the variables we use.
Thus, these parts should go into a \emph{header file}
(with extension \code{.h}),
while the actual implementation will go into an implementation
(\code{.cpp}) file.

The header file should make copious use of comments to describe what
\emph{exactly} the functions do. For instance, what happens if the
\code{toRemove} is not actually an element of the list?
What format is used for printing the list?

A moment ago, we said that the header file also contains the variables
(such as \code{head}) inside the class. While this is true, to keep our
code as modular as possible, other pieces of code should not really be
allowed to directly access those variables\footnote{One downside of
  this example is that our \code{remove} function actually works with
  a pointer to an element. So in this sense, the implementation is not
  really hidden. We will overlook this issue for now, for the sake
  of continuing with this example, but if you were bothered a bit by
  this: Way to go!}.
For instance, if a week later, we decide to use a \code{vector}
implementation to achieve the same functionality,
we do not want any code out there to rely on the existence of a variable named \code{head}. 
We want to hide as much about the class as possible.

The way to do this is to declare the variables \code{private:}
this means that only functions in the class are allowed to access
those variables, but not any other code. 
(We can also declare functions to be \code{private},
something we mostly do for internal helper functions.)
The opposite of \code{private} variables/functions are \code{public}
ones: these can be used by all other parts of the
program. (There is also a \code{protected} modifier, which we
  will learn about later, when we learn about inheritance.)
\code{private, public}, and \code{protected} are called \todef{access
  modifiers}, since they modify who can access the corresponding
elements of the class.
Our new version of the class declaration looks as follows:

\begin{verbatim}
class IntLinkedList {
  public:
    void append (int n);
    void remove (Item *toRemove);
    void print ();
  
  private:
    Item *head;
}
\end{verbatim}

In this context, it is also important to know about \code{get} and
\code{set} functions. Often, you will declare a class for which you
really kind of do want to be able to overwrite the elements.
For instance, think about \code{Item},
which we previously declared as a \code{struct}.
Instead, we could declare it as a \code{class}, as follows:

\begin{verbatim}
class Item {
  public:
    int value;
    Item *prev, *next;
}
\end{verbatim}

In order to emulate the functionality of a \code{struct} when using a
\code{class}, it is necessary to make all elements \code{public}.
But making all elements \code{public} is very bad style:
it allows other code to see the internals of a class,
which could mean that changing things in the future is harder.
But if we make the fields \code{private}, how can we read and change
the \code{value} field?
The answer is to add two functions \code{getValue} and \code{setValue},
which will read/write the \code{value} variable. 
Often, these functions can be as easy as \code{return value;} or 
\code{value = n;}.
The reason to have them is that you may later change your mind about
something. For instance, you may decide that \code{value} is a bad
name for the variable, and you want to call it \code{storedNumber}
instead. But if you do that, then all code that referenced \code{value}
will break. If you have a \code{getValue} and \code{setValue} function,
then you just have to change their implementation.

More importantly, the \code{get} and \code{set} functions allow you to
deal with illegal values, or perform other transformations.
For instance, if you decide that only positive numbers are supposed to be
stored, then you can create an error (or do something else) whenever
the given value is negative. This kind of checking of values can be
very useful in implementing something like a \code{Date} class for
storing a day of the year,
or others where you naturally want to restrict values that can be stored.

\section{Implementation of Member Functions}

Now that we have sorted out the header file, we can think about the
implementation, which we would do in a file called
\code{IntLinkedList.cpp}.
At the top, we would include the header file:

\begin{verbatim}
//IntLinkedList.cpp

#include "IntLinkedList.h"
\end{verbatim}

Then, we can start associating functions with our
\code{IntLinkedList}. The syntax for implementing class functions
is of the following form:

\begin{verbatim}
void IntLinkedList::append (int n) {
  // implementation...
}
\end{verbatim}

That is, the function name is really the class name, followed by two
colons, and then the name of the function.
This tells the compiler that this code belongs to the class. 
If we did not include \texttt{IntLinkedList::} in the function
signature, \texttt{append} would just be a simple global function. 
The compiler has no way to know that we intend our function to be part
of a class, unless we either put it inside 
\code{class IntLinkedList \{ ... \}} (which is bad style),
or put the class name in the signature, as we just did.

For the actual implementation of the functions, we can just basically
copy over the code from our previous linked list implementation --- 
the only difference is that now, the functions are member functions of
a class, rather than global functions, and the variable \code{head} is
a (private) member variable of the class rather than having to be
passed as a function parameter.
So we will not repeat all the code here. Inside the class, you can treat
all variables (even the private ones) like global variables:
all member functions know about all of the class's members.

However, before moving on, let us briefly revisit our nice recursive
\code{traverse()} function, which we could use, for instance, to print
the entire list, or print it in reverse order. Let us look at the
reverse order print version, and call it \code{printreverse()}.

\begin{verbatim}
void printreverse (Item *head)
{
  printreverse (head->next);
  cout << head->value;
}
\end{verbatim}

In this function, the \code{head} pointer was not always pointing to
the actual head of the list, but rather, we had it point to different
members of the list during different function calls.
On the other hand, the overall public signature for the function
should probably be

\begin{verbatim}
...
  public:
    void printreverse ();
...
\end{verbatim}

So how can we implement the recursive function? The answer is that we
declare a private helper function. Something like

\begin{verbatim}
...
  private:
    void _printreversehelper (Item *p);
...
\end{verbatim}

Then, the implementation of \code{printreverse} is just
\begin{verbatim}
void LinkedList::printreverse ()
{ _printreversehelper (head); }
\end{verbatim}

\section{Constructors and Destructors}
\label{sec:classes:constructors}

We could now use our implementation to create an object of type
\code{IntLinkedList} and append/remove elements from it:
\begin{verbatim}
// main.cpp
int main (void)
{
  IntLinkedList *myList = new IntLinkedList;
  for (int i = 1; i < 10; i ++) myList->append (i);
  myList->print ();
  return 0;
}
\end{verbatim}

But one concern at this point is: where does the \code{head}
variable get initialized? Earlier, we said that 
we express an empty linked list by having \code{head=NULL}.
How can we be assured that when we create a new object of type
\code{IntLinkedList}, the pointer is actually \code{NULL}?
Because we made the variables private, we cannot add the line
\code{head=NULL;} before the loop.
So perhaps, we should create a member function \code{initialize} in
the class \code{IntLinkedList}.
Then, we just have to remember to call it right after generating the object.

Fortunately, C++ has a mechanism called \todef{constructor} to do just
that, and do it automatically.
We can define \todef{constructor functions} with a special name which
get automatically called when we create a new object of that class.
Constructor functions are the right place to put initialization code.
The name of a constructor is always the name of the class,
and a constructor does not return anything. 
The constructor will be run as soon as an object is created.
It looks as follows:

\begin{verbatim}
IntLinkedList::IntLinkedList() {
  head = NULL;
}
\end{verbatim}
Notice (again) that there is no return type, not even \code{void}.
You can have multiple constructors;
for instance, we may add a constructor that initializes the list to
contain one element with a given number:

\begin{verbatim}
IntLinkedList::IntLinkedList (int n) {
  head = NULL;
  append (n);
}
\end{verbatim}

If we define two constructors, when we create the object, we can write
things like
\begin{verbatim}
IntLinkedList *p = new IntLinkedList (), *q = new IntLinkedList (3);
\end{verbatim}
The first case calls the first constructor
(and creates an empty list),
while the second calls the second constructor
(and creates a list with one item, containing the number 3).
This is often useful for copying all data from one object to another,
or reading an entire object from a file or a string.
(We will learn more about those so-called copy constructors soon.)
So long as their signatures (number or types of arguments) are
different, we can create as many constructors as we want.
Of course, they should all be declared in the header file, and
implemented in the \code{.cpp} file.

Similar to the initialization, we may also want to destroy data
objects we created when deleting an object.
Per default, when we call \code{delete myList;} in the code above,
it will free up the space used for storing the pointer \code{head},
but it will not free up any of the memory for the elements of the linked list. 
That is a good thing --- after all, other code may still need them.
So if we do want them deleted, we need to create a function that does
the ``opposite'' of the initialization.

Such a function is called a \todef{destructor},
and the destructor is automatically called when the object is deleted. 
Just as for constructors, there is a naming convention for destructors:
they always have the same name as the class, with a preceding tilde:
\code{IntLinkedList::\textasciitilde{}IntLinkedList()}.
And just like a constructor, a destructor has no return type, not even
\code{void}.
Our implementation of the destructor will look something like the
following:

\begin{verbatim}
IntLinkedList::~IntLinkedList ()
{ 
  Item *p = head, q;
  while (p != NULL)
       {
          q = p->next;
          delete p;
          p = q;
       }  
}
\end{verbatim}

\section{The \code{this} pointer}

Sometimes, in implementing a member function, you will run into
scoping issues, such as:
a function has a local variable with the same name as a member variable.
For instance, imagine that we write a \code{setValue} function for the
class \code{Item}, and it looks as follows:
\begin{verbatim}
void setValue (int value)
{ // code here
}
\end{verbatim}
We now want to somehow set the \code{value} field of the class to the
variable value, but writing \code{value=value} clearly is not going to
work, since --- whichever variable \code{value} actually refers to (the
answer is the function's parameter) --- both mentions of \code{value}
will refer to the same variable.
To make explicit that an assignment or other operation is talking
about a member of the particular object to which the function belongs,
C++ uses the keyword \code{this}. 

\code{this} is always a pointer to the object to which the method
belongs. So we can write \code{this->value} to refer to the data of the
object itself, and write the preceding code as:
\begin{verbatim}
void setValue (int value)
{ this->value = value; }
\end{verbatim}

There are other, more important, uses of the \code{this} pointer that
you will learn about, partly in this class, and more importantly as
you move on. One of the most important ones arises when you have one
object A generating another object B, and B is supposed to know A's
identity. For instance, A could be the main part of an application,
and B a part of the user interface that is being created on the fly. 
Now, A might want B to call some function in A when the user presses a
certain button. But for that, B needs to be told the ``identity'' of
A. The way one normally does this is that when B is created, it gets
passed the \code{this} pointer of A, and thus knows about A.

You can think of the \code{this} pointer as a way for an object to
know who/what it is, or more precisely, to find out where in memory it
is stored. Since most passing around of objects is done by passing
around pointers, it is very useful for an object to find out and then
pass around its own memory location.