If you've gotten this far, you are expected to be familiar with the basics of programming C++.
Consider features such as:

\begin{itemize}
\item Primitive data types: \texttt{int}, \texttt{string}, \texttt{float}, \texttt{double}, etc.
\item Composite data types: arrays, \texttt{struct}.
\item Control statements: \texttt{for}, \texttt{while}, \texttt{do while}, \texttt{if}, \texttt{else}, and \texttt{switch}.
\item Functions: declaration, iteration vs. recursion.
\end{itemize}

In principle, these features are enough to solve any programming problem.
In fact, in principle, it is enough to have nothing except for \texttt{if}, \texttt{while}, integers, and arithmetic.
Everything beyond those basics is "just" there to help us write code that is better, faster, more easily maintained, or more easily understood.

Of course, these are all fairly critical aspects of programming in the real world.
Consequently, there is great value in continuing past basic proficiency, and in order to do so, we will focus on two categories of things we can learn that will help us on our way:

\begin{enumerate}
\item Ideas, techniques, and patterns that enable us to solve problems that we didn't know how to solve (even though we had the tools to).
\item Language features and programming concepts that help us write code that is easy to read, debug, and extend.
\end{enumerate}

At this stage, you might think about programming in terms of the data your program interacts with: data is input, the program executes commands, and some corresponding output data is produced.
In this context, the sequence of instructions, i.e. the \emph{algorithm}, is the emphasis of our thought process.

The goal of this class, and by extension, this textbook, is teach you how to emphasize how data is transformed throughout the execution of a program.
We will cultivate this skill by examining new ideas, techniques, and patterns in the form of data structures, and you will learn to express them through the lens of object-oriented programming by learning new C++ language features and programming concepts.
Along the way, you will see how thinking about the organization of your data can provide insight into the efficiency of your programs.
Ultimately, this will enable you to write code that is more maintainable, easier to understand, and potentially faster.

\section{An Example}

\begin{displayquote}
Two TA's each get a list of the names and emails of the students in the class.
One gets the list sorted alphabetically, while the other gets it randomly shuffled.
Both TA's are asked to look up the email of one particular student by name.
Who would you expect to be faster at the task?
Except in rare cases, the TA with the sorted list will be much faster, and indeed, this is what happens when we run this experiment in the classroom. 
\end{displayquote}

The way we organize our data can have a profound impact on how fast (or slowly) we can solve a computational task. 
\emph{How we organize data to support the operations we need is the centerpiece of what you will learn in this class}.

What do we mean by "support the operations we need"?
Let's go back to the example of looking up emails of students by name.
Here, the key operation is clearly the "lookup" operation, at least from the standpoint of the student doing the looking up.
But when we look at the bigger picture, those data items (pairs of a student name and an email) had to be added into that form somehow.
And sometimes, we may want to delete them.
So really, the three operations we are interested in are:

\begin{itemize}
\item Insert a pair of a name and email into the list.
\item Remove a name from the list.
\item Look up the email for a given name.
\end{itemize}

Purely from the perspective of speeding up the lookup operation, the sorted list was much better than the unsorted one.
But in terms of inserting items, it is clearly better to keep the list unsorted, as we can just append items at the end of the list, and not have to worry about putting the item in the right place.
For removal---well, you will learn about that a little more later.

It is easy to imagine scenarios where keeping the list sorted is not worth it.
If we need to add data items all the time but very rarely look them up (say, some kind of backup mechanism for disasters), then sorting the list is not worth it.
Or perhaps the list is so short that even scanning through all the items is fast enough.

The answer to "what is the best way to organize my data?" is almost always "it depends."
You will need to consider how you will be interacting with your data and design an appropriate data structure \emph{in the context of its purpose}. 
Will you be searching through it often?
Will data be added in frequent, short chunks, or in occasional huge blocks?
Will you ever have to consolidate multiple versions of your data structure?
What kinds of queries will you need to ask of your data?

\section{The Notion of Abstract Data Structures}

We said that we wanted our data structure to support the operations \texttt{add}, \texttt{remove}, and \texttt{lookup}.
Let us be a little more general than just thinking about a list of students.

\begin{itemize}
\item The \texttt{add(key, value)} operation gets a \emph{key} and a \emph{value}.
In our case, these are both strings, but in general, they don't have to be.
This operation creates an association between the key and value.
\item The \texttt{remove(key)} operation gets a key and removes the pair that has the given key.
\item The \texttt{lookup(key)} operation gets a key, and returns the corresponding value, assuming the key is in the structure.
\end{itemize}

This type of data structure is typically called a \emph{map} or a \emph{dictionary}.

Is a sorted list a map?
Not exactly, although both it and the unsorted list do provide the operations we're looking for.
The word "map" describes the \emph{functionality} we want from a data structure, i.e. \texttt{add}, \texttt{remove}, and \texttt{lookup}.
On the other hand, a sorted list is a \emph{specific implementation} that happens to offer these operations.

This distinction is very important.
New programmers have a tendency to approach code by planning out their entire programmatic implementation.
As you take on bigger projects, work in larger teams, and generally learn to think more like a computer scientist, you get used to the notion of \emph{abstraction}: separating the "what" from the "how".
Using abstraction allows you to focus on big-picture design and implementation-specific details independently, which can in turn facilitate better code.

Abstraction also offers benefits in collaborative environments.
For example, you if you were working on a project with a friend, you could ask them to write certain functions without having to think through how they will be implemented.
That would be your friend's job.
So long as your friend's implementation is correct, sufficiently fast, and meets the specification that you prescribed (say, in terms of which order the parameters are in), you can simply use it as is.

So the way we would think about a map or dictionary is mostly in terms of the functions it provides: \texttt{add}, \texttt{remove}, and \texttt{lookup}.
Of course, it must store data internally in order to be able to provide these functions correctly, but how it does that is secondary---that is part of the implementation.
This combination of data and code, with a well-specified interface of functions to call, is called an \emph{abstract data type}.

One of the central goals of this course is to change the way you think about your code and how it interacts with data.
Right now, you might see a program as a simple process that gets passed some input data and processes them into output data.
Progressing through the material, you will start thinking about the data your programs interact with in terms of abstract structures with functions that help process them.
For example, an array in which you can only add things at the end is different from one in which you are allowed to overwrite everywhere, even though the actual way of storing data is the same.
This leads us naturally to \emph{object-oriented design} of programs. 

\section{Object Oriented Programming}

An \emph{object} consists of data and code; it is basically a \texttt{struct} with functions in addition to the data fields. 
This opens up a lot of ideas for developing more legible, maintainable, and "intuitive" code, as we ourselves naturally tend to think of things in the real world as discrete objects.
Different objects serve different purposes, and they interact with each other in particular ways.

Thinking in terms of object-oriented design and abstract data types can help you achieve \emph{encapsulation}.
Encapsulation describes shielding as much of the implementation of an object as possible from the outside so that it can be changed without negatively affecting the code around it.
Encapsulation helps you achieve modularity, i.e., it ensures that different pieces of code can be analyzed and tested in isolation more easily.

\section{Summary}

To summarize, what we aim to teach you in this class and textbook are:

\begin{enumerate}
\item Techniques for implementing common data structures to provide efficient functionality.
Some of these techniques will require strong fundamentals, and their analysis will enter into more mathematical territory, which is why we will be drawing on material you learned in CSCI 170.
\item How to think about code more abstractly, how to separate the "what" from the "how" in your code design, and how to utilize abstract data types to this end.
\item Good programming practice with object-oriented design, in particular as it relates to implementing data types.
\item Advanced features of C++ that facilitate these objectives.
\end{enumerate}
