If you've gotten this far, you are most likely familiar with the basics of programming C++.
Specifically, you probably have a good grasp of features such as

\begin{itemize}
	\item Primitive data types: \texttt{int}, \texttt{string}, \texttt{float}, \texttt{double}, etc.
	\item Composite data types: arrays, \texttt{struct}.
	\item Control statements: \texttt{for}, \texttt{while}, \texttt{do while}, \texttt{if}, \texttt{else}, and \texttt{switch}.
	\item Functions: iterative vs. recursive.
\end{itemize}

In principle, these features are enough to solve any programming problem.
In fact, in principle, it is enough to have nothing except \texttt{while} loops, integers, and arithmetic.
Everything that we learn beyond those basics is ``just'' there to help us write better, faster, or more easily maintained or
understood code.
Writing ``better'' code comes in two flavors in this context:
\begin{enumerate}
\item Learning problem-solving and conceptual ideas that let us solve
  problems which we didn't know how to solve (even though we had the
  tools in principle), or didn't know how to solve fast enough.
\item Learning new language features and programming concepts helps us
  write code that ``feels good,'' in the sense that it is easy to
  read, debug, and extend.
\end{enumerate}

The way you are probably thinking about programming at this
point is that there are some input data
(maybe a few data items, or an array),
and the program executes commands to produce an output from the input data.
Thus, the sequence of instructions --- the algorithm --- is at
the center of our thinking.

As you become a better programmer, it is often helpful to put the data
themselves at the center of your thinking, focusing on how data
get transformed or changed throughout the execution of the program. 
Of course, in the end, the program will still mostly do the same,
but this view often leads to much better (more maintainable, easier to
understand, and sometimes also more efficient) code. 
Along the way, you will see how thinking about the organization of your
data can be really important for determining how fast your code
executes.

\begin{example} \label{exm:overview:binary-search}
Suppose that two students each get a list of the names and e-mails of
all students in the class.
One student gets the list sorted alphabetically,
while the other's version contains the names in random order.
Both students are asked to look up the e-mail of one particular
student by name.
Who would you expect to be faster at the task?

Except in rare cases, the student with the sorted list will be much
faster, and indeed, this is what happens when we run this experiment
in the classroom. 
\end{example}

What Example~\ref{exm:overview:binary-search}
shows us is that the way we organize our data can have a profound
impact on how fast (or slowly) we can solve a computational task. 
\emph{The basic question of how to organize data to support the
  operations we need is the centerpiece of what you will learn in this
  class.}

What do we mean by ``support the operations we need?''
Let's go back to the example of looking up e-mails of students by name.
Here, the key operation is clearly the ``lookup'' operation,
at least from the standpoint of the student doing the looking up.
But when we look at the bigger picture,
those data items (pairs of a student name and an e-mail) had to be
added into that form somehow.
And sometimes, we may want to delete them.
So really, the three operations we are interested in are:

\begin{itemize}
\item Insert a pair of a name and e-mail into the list.
\item Remove a name from the list.
\item Look up the e-mail for a given name.
\end{itemize}

Purely from the perspective of speeding up the lookup operation,
the sorted list was much better than the unsorted one.
But in terms of inserting items, it is clearly better to keep the list
unsorted, as we can just append items at the end of the list, and not
have to worry about putting the item in the right place.
For removal --- well, you will learn about that a little more later.

It is easy to imagine scenarios where keeping the list sorted is not
worth it.
For instance, if we need to add data items all the time,
but very rarely look them up (say, some kind of backup mechanism for
disasters), then sorting the list is not worth it.
Or maybe, the list is so short that even scanning through all the
items is fast enough.

The main take-away message is that the answer to ``What is the best
way to organize my data?'' is almost always ``It depends.'' 
You will need to consider how you will be interacting with your data,
and design an appropriate data structure \emph{in the context of its purpose}. 
Will you be searching through it often?
Will data be added in frequent, short chunks, or in occasional huge blocks?
Will you ever have to consolidate multiple versions of your data structure?
What kinds of queries will you need to ask of your data?

\medskip

There is another interesting observation we can take away from the previous
example. We said that we wanted our data structure to support
the operations \texttt{add}, \texttt{remove}, and \texttt{lookup}.
Let us be a little more general than just thinking about a list of students.
\begin{itemize}
\item The \texttt{add (key, value)} operation gets a \emph{key} and a
  \emph{value}. (In our case, they were both strings, but in general,
  they don't have to be.)
  It creates an association between the key and value.
\item The \texttt{remove (key)} operation gets a key, and removes the
  pair that has the given key.
\item The \texttt{lookup (key)} operation gets a key, and returns the
  corresponding value, assuming the key is in the structure.
\end{itemize}
This type of data structure is typically called a \emph{map},
or a \emph{dictionary}.

So is a sorted list a map? Not exactly.
After all, an unsorted list would also be a map.
There seems to be a difference.
The main difference is that the word ``map'' describes the
\emph{functionality} that we want:
\emph{what} the data structure is supposed to do.
On the other hand, a sorted list (or an unsorted list) is a
\emph{specific way} of getting this functionality:
it says \emph{how} we do it. 

This distinction is very important. When you start to program,
usually, you just go straight ahead and plan out all of your code.
As you take on bigger projects, work in larger teams, and generally
learn to think more like a computer scientist, you get used to the 
notion of \emph{abstraction}: separating the ``what'' from the ``how.''
You can work on a project with a friend, and tell your friend
which functions you need from him/her, without having to think through
how they will be implemented. That is your friend's job.
So long as your friend's implementation works (fast enough) and meets
the specification that you prescribed (say, in terms of which order
the parameters are in), you can just use it as is.

So the way we would think about a map or dictionary is mostly in terms
of the functions it provides:
\texttt{add}, \texttt{remove}, and \texttt{lookup}.
Of course, it must store data internally in order to be
able to provide these functions correctly, but how it does that is
secondary --- that is part of the implementation.
This combination of data and code, with a well-specified interface of
functions to call, is called an \emph{abstract data type}.

Once you start realizing the important role that data organization
(such as using abstract data types) plays in good code design,
you may change the way you think about code interacting with data.
When starting to program, you probably thought about code that gets
passed some data (arrays, structs, etc.) and processes them.
Instead, you may now think of the data structure itself as having functions
that help with processing the data in it. 
In this way of thinking, an array in which you can only add things at
the end is different from one in which you are allowed to overwrite
everywhere, even though the actual way of storing data is the same.

This leads us naturally to \emph{object-oriented design} of programs. 
An \emph{object} consists of data and code; it is basically a
\texttt{struct} with functions inside in addition to the data fields. 
This opens up a lot of ideas for developing more legible,
maintainable, and ``intuitive'' code.
Once you think about objects, they often become almost like little
people in your mind.
Different objects serve different purposes,
and they interact with each other in particular ways.

One of the main things that thinking in terms of object-oriented
design (and abstract data types) does for you is to help achieve 
\emph{encapsulation}: shielding as much of the inside of an object as
possible from the outside, so that it can be changed without
negatively affecting the code around it.
The textbook refers to this as ``walls'' around data.
Encapsulation helps you achieve modularity, i.e., it ensures that
different pieces of code can be analyzed and tested in isolation more 
easily.

\medskip

To summarize, the learning goals of this class are:
\begin{enumerate}
\item Learn the basic and advanced techniques for actually
  implementing data structures to provide efficient functionality.
  Some of these techniques will require strong fundamentals,
  and their analysis will enter into more mathematical territory,
  which is why we will be drawing on material you will be learning
  simultaneously in CSCI 170.
\item Learn how to think about code more abstractly, separating the
  ``what'' from the ``how'' in your code design and utilizing abstract
  data types to specify functionality.
\item Learn about good programming practice with object-oriented
  design, in particular as it relates to implementing datatypes.
\end{enumerate}
