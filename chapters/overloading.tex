In this chapter, we will first learn about overloading and operators,
then see how to combine them to create easier-to-read code.
We will also see where and how copy constructors are used,
and learn about the difference between shallow and deep copying.

\section{Overloading and Operators}
Function overloading simply means having multiple functions of the
same name, but with different numbers or types of parameters.
For instance, when we write something like:
\begin{verbatim}
int minimum (int x, int y);
double minimum (double x, double y);
\end{verbatim}
we have overloaded the \code{minimum} function.
Doing this is perfectly acceptable:
when you write \code{cout << minimum (a, b)},
the compiler can tell from the arguments which of the two function to use.
It is only when you try to have two functions with the same exact
signature that you run into trouble:
\begin{verbatim}
bool foo (int x);
void foo (int y);
\end{verbatim}
Now, if you call \code{foo(a)} on an integer \code{a},
the compiler would not know which of the two functions you mean,
so it will not let you write this code in the first place. 
To summarize, \todef{overloading} is just having multiple functions of the same name.

\todef{Operators} are symbols such as 
\code{==}, \code{!=}, \code{=} (assignment), \code{<=}, \code{+},
\code{-}, \code{++}, \code{*} (multiplication, and dereferencing), \code{[]}, 
\code{<<}, \code{>>}, and many others.
Most operators are binary, that is, they take two arguments, such as
\code{x==y}, \code{x=y}, \code{a+b}, \code{a[i]}.
But there are also some unary ones (taking one argument),
such as \code{!b}, \code{*x}, \code{i++}.
For operators, C++ knows where the arguments go. For many operators
(\code{=}, \code{==}, \code{<=}, \code{+}, \code{<<}),
one argument goes before the operators, the other after.
But for instance, for \code{a[i]}, one argument goes ``in the middle''
of the operator.

Using operators is really just shorthand for calling a function.
When you write the following code:
\begin{verbatim}
    if ( a1 == a2 ) cout << "They are equal!";
\end{verbatim}
it actually gets translated to
\begin{verbatim}
    if (operator==(a1, a2)) operator<< (cout, "They are equal!");
\end{verbatim}
If you wanted, you could directly write the second version in your code.
The reason you should not do so is that it would be much harder to read.
In a sense, operator overloading is just what some people call
``syntactic sugar:'' parts of the programming language that do not
really add functionality,
or even necessarily good coding practices (like code reuse),
but just make your code look prettier. 
For instance, suppose that \code{x,y} are variables of some class you
have written yourself. Something like 
\begin{verbatim}
if (x > 0) y++;
   else if (x == 0) y = 0;
        else y --;
\end{verbatim}
is much easier to read than
\begin{verbatim}
if (isGreater(x,0)) addOne(y);
   else if (equals(x,0)) set(y,0);
        else subtractOne(y);
\end{verbatim}
or
\begin{verbatim}
if (x.isGreater(0)) y.addOne();
   else if (x.equals(0)) y.setEqual(0);
        else y.subtractOne();
\end{verbatim}

The ``long version'' of operator names
(such as \code{operator== (a1, a2)} in our earlier example)
does tell you how to overload operators in general:
all you need to do is to overload a function whose name is the word
\code{operator}, followed by the operator itself.
For instance, if you want to overload the \code{*} operator to allow
``multiplying'' a number by a string
(and define it to mean concatenating that many copies of the string),
you can do that as follows:
\begin{verbatim}
string operator* (int n, string x)
{
  string s;
  for (int i = 0; i < n; ++i) s.append (x);
  return s;
}
\end{verbatim}

You could now write something like
\code{cout << 5 * "Hello"} 
or \code{cout << operator* (5, "Hello")}.

\section{The \code{const} keyword}
The \code{const} keyword is not directly related to operator
overloading or copy constructors, but this is the context where it
makes a lot of sense and becomes important for the first time in this
class, so we have a discussion here.

When you initially learned about passing parameters to functions by
reference (with `\&`), the purpose was probably always so that the
function could modify the content of the variable,
and the modification would be seen by whichever other function had called it.
But there is another reason to pass parameters by reference: 
when you pass by value, a copy of your parameter is made,
which involves calling the copy constructor 
(see Section~\ref{sec:overloading:copy-constructors}).
There are some dangers to this:
if no copy constructor is implemented, the default is to make a shallow copy,
and in some cases, that is not what you want to happen. 
And if you do make a deep copy, that can be quite inefficient at times.
To avoid ambiguities like this,
when the types of your parameters are more complex
(like \code{vector} or our own \code{ArrayList}, or later on in
Chapter~\ref{chap:templates} generic/template types),
it is often easiest to just pass by reference.

But if you pass parameters by reference,
then someone using your function might suspect that you intend to
alter the object that is passed in, 
even when that is not your intention. 
The solution is to use the \code{const} keyword;
not only does it \emph{tell} the user of your class that the function
does not alter its argument,
but the compiler will \emph{ensure} that you cannot alter it. 
For instance, if you wanted to write a function \code{append} that
appends one \code{ArrayList} at the end of another,
its syntax would be as follows:

\begin{verbatim}
void ArrayList::append (const ArrayList & a2) {
    //code
}
\end{verbatim}
By adding \code{const}, you ensure that the code inside the function
can not alter the value of \code{a2} or any of its fields. 
If you tried to include code that alters \code{a2},
the compiler would return an error.%
\footnote{The exact meaning of ``altering'' \code{a2} is a bit subtle.
  To be precise, when you use the \code{const} keyword,
  you cannot make assignments to the member variables of \code{a2}.
  But if \code{a2} has a field that is a pointer
  (for instance, the internal array for the \code{ArrayList}),
  we \emph{are} allowed to change what is stored at the memory
  locations that the pointer points to.
  This may at first seem unintuitive.}
The main point of \code{const} is to make it clear to others using the
function(s) what can and cannot happen to the parameters.

Suppose that you are now implementing the \code{append} function,
and as part of your implementation,
you want to call some other member function on \code{a2}. 
How is the compiler supposed to know that this function call will not
suddenly result in changes to the member fields of \code{a2}?
This is where the second use of \code{const} comes in.
The syntax is like in the following example:

\begin{verbatim}
ArrayList::print () const {
    //code
}
\end{verbatim}
Say that we are defining a function to print the \code{ArrayList},
which should not need to alter any member fields of the object.
The \code{const} keyword now says that the function
\code{printlist} will not change the member variables of the object on
which it is called, that is, after it executes, the object will be in
the same state. 

There is a third use of the \code{const} keyword.
Sometimes, it is useful for a function to return its result by
reference.
This is typically for the same (or similar) reasons as those which
make you want to pass arguments by reference:
(1) you want to allow the code that calls the function to modify the
returned value, or
(2) the  returned value is a complex object, and you do not want a copy
constructor to be called for it.
In the second case, you again want to make it clear that the returned
object can/should not be altered.
You do this by returning a \code{const} reference.

All three uses are illustrated with the following example of the
\code{get} function of a \code{map} implementation:
% DK: the get function for maps has not really been defined here

\begin{verbatim}
const ValueType & Map::get(const KeyType & key) const;
\end{verbatim}

Because the key could be a complicated object,
you want to pass it in by reference, but the \code{get} function
should not modify it, so the key is passed in by \code{const} reference.
The \code{get} function should also not modify the state of the map,
so the whole function is declared \code{const}.
Finally, the value associated with the key could be a complex object,
so it is returned by reference;
but unless we want someone to be able to modify it
(and thus implicitly modify the map),
we should return it by \code{const} reference.
Notice that there are three objects around here:
the key, the value, and the map.
Each \code{const} refers to a different one among those three objects that
may not be altered.

Use the following as a guideline for when to mark things \code{const}
and when to pass by reference:
\begin{itemize}
\item If a member function does not need to alter the object on which
  it runs, then mark it \code{const}. It cannot hurt to do so,
  and might help you, because you can now call this function from within a
  context where the object is \code{const}.
\item If you are passing around complicated objects, unless you see a
  good reason to pass them by value, pass them by reference.
\item If you are passing an object by reference only for this reason,
  but not because the function you are writing is supposed to modify it,
  then pass it by \code{const} reference.
\item If you are returning an object by reference just because it is
  complicated, but not because you want to be able to alter it,
  then return it by \code{const} reference.
\end{itemize}

One more thing is worth mentioning about the \code{const} keyword,
because it often confuses students:
While the \code{const} keyword, applied to a member function,
prevents you from changing \emph{member} variables of the object
inside the function,
it of course does not prevent you from changing local variables of the
function, since they are not part of the object's state. 

\section{Operator Overloading and Classes}
By far the most frequent case for operator overloading is in the
context of defining new classes.
For instance, if you define your own class for complex numbers (as
we will do below), you would want to overload the arithmetic operations,
equality test, assignment, and a few others.
And for almost any class you define, you will likely want to overload
the assignment operator, the equality test \code{==}, and frequently
also \code{<<} so that you can easily print it to the screen or a file,
and do not have to write separate \code{print()} functions to
print to \code{cout, cerr} and files.

The syntax for operator overloading in classes is as follows.
Suppose that we have declared a class \code{IntArray},
which stores the size of the array and an array of integers internally.
We now want to add some operators:
\begin{verbatim}
class IntArray {
     // for binary operators
     // pattern: [RETURNTYPE] operator[OPERATOR] ([RHS DATA])
     // Example:
     bool operator== (const IntArray& otherArray) const {         
        if (this->size != otherArray.size) return false;
        else (for int i = 0; i < size; ++i)
             if (this->data[i] != otherArray.data[i]) return false;
        return true;
     }

     int & operator[] (int index) {
        return data[index];
     }

     // for unary operators
     // pattern: [RETURNTYPE] operator[OPERATOR]
     // Example:
     IntArray & operator++ () {
         // increment all entries in the data field.
         for (int i = 0; i < size; ++i) data[i] ++;
         return *this;
     }

     private:
       int size;
       int *data;
}
\end{verbatim}

We can now write things like
\begin{verbatim}
IntArray firstArray, secondArray;
// some computation to put data in the variables.
if (firstArray == secondArray) ++firstArray;
secondArray[0] = 0;
\end{verbatim}
as shorthand for
\begin{verbatim}
IntArray firstArray, secondArray;
// some computation to put data in the variables.
if (firstArray.operator==(secondArray)) firstArray.operator++;
secondArray.operator[](0) = 0;
\end{verbatim}

Two small notes: first, why do we have \code{++} return \code{*this}?
Could we have made the operator's return type void?
We could, but then we could not write things like \code{cout << ++a},
because the type of \code{++a} would be \code{void}.
Second, why did \code{operator[]} return its result by reference?
The reason is that we want to be able to overwrite it,
as we do in the example.
If we returned it by value, then overwriting would be impossible.

One type of operator we frequently overload is the assignment operator,
since we routinely want to write something like \code{x=y} for objects
of our own newly defined class.
A few interesting issues come up when you define assignment operators.
Let us try to write one for our \code{class IntArray} above:

\begin{verbatim}
IntArray & operator= (const IntArray & otherArray) {
    this->size = otherArray.size;
    this->data = new int[this->size];
    for (int i = 0; i < this->size; ++i)
        this->data[i] = otherArray.data[i];
    return *this;
}
\end{verbatim}

We just copied over all of the data from \code{otherArray}.
Why did we make the assignment operator's return type
\code{IntArray\&} instead of \code{void}?
The reason is the same as for the \code{++} operator earlier:
it allows us to write a statement such as \code{a = b = c},
which we routinely do with other types (like integers),
and would therefore like to do for \code{IntArray} as well.
For that reason, when we write \code{b=c},
what is returned must also be of the type that can be
assigned to \code{a}.

However, there is something else we should think about with the code
for the assignment operator.
Our class \code{IntArray} has dynamically allocated data
(here, an internal array), so when we write 
\code{this->data = new int[this->size]}, we create a memory leak.
We have to deallocate the old \code{data} object before making this
assignment.
So suppose that we added the statement \code{delete [] data}
as the first line of our operator. Are we good now?
Not quite. Think about what happens when we write the statement
\code{a = a}.
While this is not necessarily a very useful statement to write,
you would not want it to break someone else's code because
of \emph{your} wrong implementation of the operator. 
What happens now is that first, you delete all the data inside
\code{a}, then try to copy that (deleted) stuff into \code{a}.
So you will just copy garbage, and likely get a segfault
(if you are lucky) or really strange behavior.
So you must first check that you are not assigning an object to itself;
once you have ensured that, it is safe to deallocate the old data in
the object you are assigning to,
then copy over the data from the other object.
This means we should also add a first line
\code{if (this != \&otherArray)}.

\section{Friend Access}
Sticking with our integer array example, suppose that we wanted to
implement an operator that multiplies all entries of the array with
the same integer. We could do that as follows:
\begin{verbatim}
class IntArray { 
  // other stuff
  IntArray operator* const (int multiplier)
    { 
       IntArray newArray;
       newArray.size = size;
       newArray.data = new int[size];
       for (int i = 0; i < size; ++ i)
           newArray.data[i] = data[i] * multiplier;
       return newArray;
    }
}
\end{verbatim}
Now, we can write code like \code{a*7} to multiply the entire object
of type \code{IntArray} by 7. 
(Again, this is shorthand for \code{a.operator*(7)}.)
But what if we also wanted to write \code{7*a}?
That would be shorthand for \code{7.operator*(a)},
meaning that we would have to make our operator a member function of \code{int},
which we are not allowed to do.
So in this case, we will have to revert to the earlier method we saw
of defining the operator not as a member function of \code{IntArray}:

\begin{verbatim}
IntArray operator* (int multiplier, const IntArray & oldArray)
  { // same implementation }
\end{verbatim}

But now, we have another problem:
our implementation accessed private data fields,
and methods that are not member functions are not allowed to do that.
(For the purpose of this discussion, we ignore the fact that we could
instead use our assignment operator and the access to individual
elements using \code{[]}. We are trying to learn something new here.)
What can we do? We could just make the variables public,
but that would completely defeat the idea of hiding variables that
others do not need to see.
What we would really like to do is say that these variables are still
private, but we will make an exception just for this one particular
function \code{operator*}.
C++ has a mechanism for doing this:
you declare a function a \code{friend} of a class, as follows:

\begin{verbatim}
class IntArray {
   // other stuff
   friend IntArray operator* (int multiplier, const IntArray & oldArray);
}
\end{verbatim}

If we put this line in the definition of the class \code{IntArray},
we can then write the exact code as above outside the class,
and it does get access to private variables.
Notice that the \code{friend} declaration had to be inside the class
\code{IntArray}.
That makes sense: only the class itself should be allowed to provide
the privilege of accessing its private variables --- a function cannot
just get access to private variables by declaring itself a friend of
the class.


Multiplying an array by a number is nice, but it is not a
functionality that is often needed.
The most frequent use of \code{friend} functions is for implementing
the \code{<<} operator for output streams.
The way you do this is as follows:

\begin{verbatim}
class IntArray {
  // other stuff
  friend ostream & operator<< (ostream & o, const IntArray & a);
}

ostream & operator<< (ostream & o, const IntArray & a)
{ 
  for (int i = 0; i < a.size; ++ i)
      o << a.data[i] << " ";
  o << "\n";
  return o;
}
\end{verbatim}

Since we wanted to access private data fields of \code{a},
we declared \code{<<} a friend function of \code{IntArray}.
Also, by now you should have gotten used to the idea of why we return
an \code{ostream} here: it is so that we can write
\code{cout << a << "Hello World"}
with multiple \code{<<}.

\subsection{Friend Classes}
The following is not directly related to operator overloading,
but since we are talking about \code{friend} functions,
we should also talk about \code{friend} classes.
Sometimes, you want to allow not just one function,
but all functions of class X to access all private members of class A.
To do so, you can simply declare X a \code{friend} class of A. 
Just like with friend functions, the \code{friend} relationship must
of course be declared in class A, not in X. 
The code for declaring class X a friend of A is:

\begin{verbatim}
class A {
  public:
    friend class X;
  private:
    ...
}
\end{verbatim}

This piece of code would ensure that any object of type X would be
allowed to access all the private parts of any object of type A.

\section{An Illustration: Complex Numbers}
The following illustration shows in more detail how to actually code
with operator overloading 
(whereas the preceding discussion was perhaps a bit abstract). 

We want to implement a class for complex numbers.
Recall that complex numbers are of the form $a+\imath b$, where $a,b$ are
real numbers, and $\imath$ is such that $\imath^2 = -1$.
The addition of complex numbers is defined as expected:
$(a+\imath b) + (c+\imath d) = (a+c)+\imath (b+d)$.

We can use the following definition of a very basic \code{Complex}
class: 

\begin{verbatim}
class Complex {
  private:
    double re, im;	
  public:
    Complex (double re, double im) {
      this->re = re;
      this->im = im;
    }

    Complex (const Complex & toCopy) {
      // we add a copy constructor while we are at it.
      // See the next section to learn about copy constructors.
      this->re = toCopy->re;
      this->im = toCopy->im;
    }

    Complex add (Complex other) const {
      double reSum = re+other.re;
      double imSum = im+other.im;
      return Complex(reSum, imSum);
    }

    string toString() const {
      return (""+re+ "+" +im+"i");
    }
}
\end{verbatim}

Now, we could add two \code{Complex} numbers and print the sum using
the following code:
\begin{verbatim}
  Complex c1 = Complex(1.5,-3.2);
  Complex c2 = Complex(1.1,1.3);
  Complex sum = c1.add(c2);
  cout << sum.toString() << endl;
\end{verbatim}
	
While this works well, it is a little tedious to have to write
function calls like \code{add()} and \code{toString()}. 
This makes comprehending the code more difficult than it needs to be. 
To get around this, we can use the idea of operator overloading we
just saw.
Instead of (or in addition to) having a function \code{add()},
we overload the operator \code{+}, as follows:
\begin{verbatim}
  Complex operator+ (const Complex& other) const {
      return add(*this, other);
}
\end{verbatim}
		
Now, we can perform the third line of the code above using the
shortcut: 
\begin{verbatim}
   Complex sum = c1+c2;
\end{verbatim}

This is exactly equivalent to the previous segment,
but is much easier to interpret.
We overload a few more commonly used operators, to get a hang of it.

\begin{verbatim}
Complex operator- (const Complex & other) const {
   double reDiff = re - other.re;
   double imDiff = im - other.im;
   return Complex(reDiff, imDiff);
}

Complex operator* (const Complex & other) const {
   // look up multiplication of complex numbers if you do not remember it.
   double reProd = re*other.re - im*other.im;
   double imProd = re*other.im + im*other.re;
   return Complex(reProd, imProd);
}

bool operator==(const Complex & other) const {
   return (re==other.re && im== other.im);
}

bool operator!=(const Complex & other) const {
   return !(*this == other);
}

bool operator< (const Complex & other) const {
   double absSq = re*re+im*im;
   double otherAbsSq = other.re*other.re + other.im*other.im;
   return (absSq < otherAbsSq);
}

bool operator<= (const Complex & other) const {
   double absSq = re*re+im*im;
   double otherAbsSq = other.re*other.re + other.im*other.im;
   return (absSq <= otherAbsSq);
}

bool operator> (const Complex & other) const {
   return !(*this<=other);
}

bool operator>=(const Complex & other) const {
   return !(*this<other);
}
\end{verbatim}

With these definitions in place, we could now do arithmetic and
comparisons on our nice \code{Complex} class. 
In order to also be able to print them,
we will overload the operator \code{<<}.
Suppose that we did not already have the \code{toString()} function.
Then, to stream out the complex number,
we write the following simple function.

\begin{verbatim}
ostream & operator << (ostream &out, const Complex & c) {
    out << (c.re+ "+" +c.im+"i");
}
\end{verbatim}

Since we are accessing the private member variables,
we need to make this operator a \code{friend} function of the class
\code{Complex}, by putting the line

\begin{verbatim}
    friend otsream & operator<< (ostream &out, const Complex & c); 
\end{verbatim}

into the definition of \code{class Complex}.
We can now write code including multiple \code{<<} 
in one statement, as follows: 

\begin{verbatim}
  Complex c1 = Complex(1,2);
  Complex c2 = Complex(5,7);
  cout << c1 << " + " << c2 << " = " << c1+c2 << endl;
\end{verbatim}

\section{Copy Constructors}
\label{sec:overloading:copy-constructors}
When we started talking about classes and objects,
we also mentioned constructors briefly.
It is worth mentioning explicitly now that just like any other
functions, constructors can be overloaded; i.e.,
a single class can have many different constructors,
so long as their signatures (number/types of arguments) are different.

One type of constructor that is almost always very useful is a copy
constructor, which gets another object of the same type,
and ``copies'' its contents in some way.
Its signature will usually be:

\begin{verbatim}
class A {
   A (const A & otherA);
}
\end{verbatim}

With a copy constructor, we can ``copy'' objects as follows:

\begin{verbatim}
    A *a1;
    // some code to put stuff in a1.
    A *a2 = new A(a1);
    // a2 now contains a copy of a1.
\end{verbatim}

How much should a copy constructor copy?
That is quite an interesting question, and depends on context. 
One approach is to copy exactly the data fields of your object.
This is called a \todef{shallow copy} and is what C++ produces as a
default copy constructor if you do not write one yourself.
However, sometimes, that is not what you want.
For instance, in our example above of an integer array \code{IntArray},
a shallow constructor would copy the \code{size} and \code{data} variables.
Since \code{data} is a pointer, we would now have two objects whose
\code{data} pointers point to the same memory.
If we make an edit in one, it will edit the other.
Perhaps that is what we want, and perhaps not.
If it is not what we want, then what we wanted was likely to copy all
of the contents of the array to a new location as well.
In that case, we would be making a \todef{deep copy},
and we have to write a copy constructor to do it correctly.
Deep copies can involve a bit of work:
for instance, for a linked list, you would have to make a copy of
each element of the list, and link them all up correctly as you go along.
Often, that is exactly what you want from a copy.
So make sure you know exactly what you want to copy and why.
Otherwise, you may get surprising results.

To illustrate what can go wrong, suppose that for our \code{IntArray} above,
we make a shallow copy, with the following implementation:
\begin{verbatim}
class IntArray {
  // other stuff
  IntArray (const IntArray & otherArray)
    { size = otherArray.size; data = otherArray.data; }
  ~IntArray ()
    { delete [] data; data = NULL; }
}

int main (void)
{
  IntArray a1;
  IntArray a2 (a1);
  return 0;
}
\end{verbatim}

This program will segfault.
What happens is that \code{a2} will contain shallow copy of \code{a1}.
When the program terminates, the destructors of both objects are
called, to give them a chance to clean up their affairs before going
away (e.g., the objects may want to store some data on disk).
Suppose that \code{a1} gets destroyed first.
Then, its \code{data} gets deleted, and is deallocated. 
But the pointer \code{a2.data} still points to the location,
and when \code{a2} gets destroyed,
it tries to delete memory that has already been deallocated.
When you debug this program, you would wonder why it segfaults in the
line \code{return 0}, which seems rather innocuous.
This type of subtle errors can show up in unexpected parts of your
program, and is not infrequently caused by the wrong type of copy
constructor.  

Why are copy constructors so important?
To answer that question, let us return to our earlier discussion of
passing parameters by value vs.~passing them by reference.
You recall that in this class, we have been doing much of our passing
of parameters by reference,
and then declaring them \code{const} to avoid them getting altered.
So far, we have been a bit vague about ``passing around big objects,''
and ``it is not clear what happens when you pass a complicated object''
into a function as a parameter by value.
Actually, it \emph{is} clear what happens.
C++ generates code that initializes your local variable using a copy
constructor. So when you have the following code:
\begin{verbatim}
int foo (A myAObject)
  { ... }
\end{verbatim}
and you call \code{foo (b)}, what will happen is that \code{myAObject}
gets initialized via copy constructor from \code{b}.

Exactly the same thing applies when you have a function that returns
by value vs.~one that returns by reference.
When you return by value, the variable that you assign to
(or the place where the return value is used) gets initialized by copy 
constructor from what you return.

In both of the preceding cases, 
if the class \code{A} has a copy constructor that does exactly what
you want, passing complex objects by value is perfectly fine,
and your code should execute exactly as you expected
(except it may be a bit inefficient to copy so many objects). 
But if \code{A} makes a shallow copy when you needed a deep copy,
or makes a deep copy when you needed a shallow one,
then you could get very unexpected results.
In particular, because \code{myAObject} is a local variable,
it will be destroyed when \code{foo} ends.
If \code{myAObject} were a shallow copy of --- say --- an
\code{IntArray}, then the \code{data} of that \code{IntArray} would be
deallocated.
This would likely have terrible consequences elsewhere in the program.
And remember: when you have not explicitly defined a copy constructor,
C++ will generate a default one which performs a shallow copy.

This type of issue is even more pronounced when you write
templated/generic classes (as you will learn in Chapter~\ref{chap:templates}).
In that case, sticking with our previous example, you would not even
know what the class \code{A} is.
Some users may use your function \code{foo} with classes that have a
deep copy constructor,
others with classes that have a shallow copy constructor. 
Unless you are very sure that your function works equally well in both cases,
it is much safer to simply pass parameters by reference,
and side-step the use of a copy constructor altohgether.
And that is why we have been advocating passing complex parameters by reference.

\subsection{The Rule of Three}
Now that we have discussed in depth what a copy constructor does,
and when you should write one, we should ask:
what about the assignment operator?
Does it also need to be written?
The answer is typically that if you need to write your own copy
constructor (because you do not want the default shallow copy),
then you will virtually always also want your own assignment operator.
The reason is that when you need a deep copy from your copy constructor,
then a deep copy is probably the ``right'' way to copy objects of your
class. 
Then, an assignment such as \code{a = b} should probably also make a
deep copy, as it would be very surprising for a user that assignment
does not do the ``right'' thing.

And for that matter, if you need a deep copy, it is virtually always
because there is some dynamic memory allocated inside your object.
Then, you should also have a destructor which deallocates that memory.
The rule of thumb that we can derive is then the following (sometimes
called the ``Rule of Three''): If you implement one out of the
following three, then you almost always need to implement all three of
them:

\begin{itemize}
\item Copy constructor
\item Assignment operator
\item Destructor
\end{itemize}
