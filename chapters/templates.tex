Once we have carefully implemented a class like \code{List} or
\code{Stack} or \code{Queue}, we may discover that it is quite nice,
and we would like to also have a List or Stack or Queue of strings.
And one for users, and for web pages, and so many other types of objects. 
The obvious solution would be to copy the code we wrote for integers,
and then replace the word \code{int} with \code{string} everywhere.
This will cause a lot of code duplication.
In particular, imagine now that after making 20 different copies, we
discover that the original implementation of \code{List}
actually had a few bugs. Now we have to fix them in all copies.

What we'd really want is a mechanism for creating a ``generic''
\code{List} class, which allows us to substitute in any type of
data, rather than just \code{int}, or just \code{string}.
This way, we avoid code duplication and all the problems it creates
(in particular in terms of debugging and keeping versions in synch).

The C++ mechanism for doing this is called \todef{templates}. 
You've already used this when using the C++ STL (Standard Template
Library), and even earlier when using \code{Vector<...>}.
The syntax for creating your own template classes is a little clunky,
and because of the particular way in which C++ implements it, it
unfortunately also forces us to give up some of our good coding practices. 
(It was not originally part of the C++ design, but rather grafted on
top later, which explains why it does not fit very smoothly.)

The basic way of defining templates classes is explained first via an
implementation of a generic \code{struct Item}: 
\begin{verbatim}
template <class T>
struct Item
{
    T data;
    Item<T> *prev, *next;
}
\end{verbatim}

This tells C++ that we are defining a template class, which will be
generally based on another class. We call that class \code{T} for now.
By writing this, we can treat \code{T} as though it were an actual
name of an existing class.

What really happens is the following. 
Suppose that in some later code, we want to create an
\code{Item} for strings. We will now write:

\begin{verbatim}
Item<string> *it = new Item<string>;
\end{verbatim}

When the compiler sees this code, it will copy over the template code,
and wherever it sees \code{T}, it will instead substitute
\code{string}. In other words, it will exactly do the text replacement
that we wanted to avoid doing by hand. Of course, in this way, we only
maintain one piece of code, and the rest is generated only at compile
time, and automatically. That's how templates save us work.

We can now apply the same ideas to our LinkedList class:

\begin{verbatim}
template <class T>
class LinkedList {
  public:
    LinkedList ();
    ~LinkedList ();
    void append (T value);
    void remove (Item<T> *toRemove);
    void printlist ();
  
  private:
    Item<T> *head;
}
\end{verbatim}
   
When we implement class functions of template classes, the syntax is
as follows:

\begin{verbatim}
template <class T>
LinkedList<T>::add (T item) {
     //code
}
\end{verbatim}

Note the \code{template<class T>} line and the
\code{<T>} in the class name. 
You have to put these in front of \emph{each} member function
definition; otherwise, how is the compiler to know that you are
implementing a member function of the template class?
Usually, if you get a ``template undefined'' error, you might have
missed a \code{<T>}.
(Note that the identifier \code{T} is arbitrary ---
you can use \code{X} or \code{Type} or whatever). 

By the way, the keyword \code{template} cannot be used just with
classes, but also with other things, such as functions.
For instance, we could write the following:

\begin{verbatim}
template <class Y>
Y minimum (Y a, Y b)
{ if (a < b) return a; else return b; }
\end{verbatim}

Another thing worth noting about templates is that you can't just use
one type: you can use multiple types. For instance, remember
that a \code{map} has two types: one for the key, and one for the
value. We could specify that as follows:

\begin{verbatim}
template <class keyType, class valueType>
class map
{
  // prototypes of functions and data go here
}
\end{verbatim}

Templates were kind of grafted on top of C++, and not really part of
the initial design. That is why the syntax is rather unpleasant, and
the ``obvious'' way of doing things causes all kinds of compiler and
linker problems. In order to get things to work, you will sometimes
need to deviate from good coding practices, for instance by having 
\code{\#include} of a \code{.cpp} file.
Check the course coding guide for detailed information
on how to deal with some of the most common problems.
Seriously, read it very carefully before attempting to code anything
with templates. It will save you many hours of debugging - promised!

\section{Some other considerations}
Earlier on, in Chapter~\ref{chap:overloading}, we learned about friend
functions. Friend functions require some extra care when you combine
them with defining your own template classes.

When the function you want to declare as a friend takes a template
parameter (because your class is templated), you need to say so inside
the class definition, and give it a different symbolic name to avoid
ambiguity. 

For instance, suppose that you want to implement a generic
``multiplication'' of an integer with an array --- maybe it
concatenates the array with itself $n$ times.
So you'd need to overload the \code{*} operator and declare that
operator a friend of your \code{ArrayList} class.

\begin{verbatim}
template <class T>
class ArrayList { 
  // other stuff
  template <class TT>
    friend ArrayList<TT> operator* (int n, const ArrayList<TT> & oldArray);
}
\end{verbatim}

Notice that we named the other template parameter something other than
that \code{T}; here, we used \code{TT}.

