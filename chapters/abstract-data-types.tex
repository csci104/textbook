If we take a step back from what we have been doing with Linked Lists,
we can think about the functionality they enable.
Let us look at the version we just analyzed,
which allows us to do three things:
\begin{enumerate}
\item Add an integer $n$ to the list.
\item Remove an item from the list.
\item Print all the integers in the list, in the order they were added.
\end{enumerate}

If all we care about is being able to perform these three operations,
it does not matter so much whether we implement them using linked
lists, or perhaps arrays, or vectors, or some other method. 
What we really want is some data structure that stores data
internally, and allows us to use these three operations. 
In other words, we focus on \emph{what} we want to do,
not \emph{how} exactly it is done.
Linked lists were one answer to \emph{how}.

Being precise about the \emph{what},
i.e., the operations we want to see implemented, or are implementing,
is what is specified as an \todef{Abstract Data Type (ADT)}. 
An ADT is defined entirely by the operations it supports,
such as the ones above.
In this class, we will spend a lot of time focusing on the following three ADTs.

\begin{description}
\item[List:] A list data type is defined by supporting the following
  key operations (where \code{T} denotes any type, such as \code{int},
  \code{string}, or something else):
\begin{enumerate}
\item \code{insert (int position, T value)}:
  inserts the value right before the given position,
  moving all the later elements one position to the right.
\item \code{remove (int position)}:
  removes the value at the given position,
  moving all the later elements one position to the left.
\item \code{set (int position, T value)}:
  overwrites the given position with the given value.
\item \code{T get (int position)}:
  returns the value at the given position.
\end{enumerate}

Thus, the \code{List} data type is a lot like an array in its functionality,
but it allows inserting and removing elements while shifting all others.

Notice that this definition is not quite complete.
For instance, it does not tell us what values of \code{position} are legal,
or what happens when they are out of range.
For instance, if we are trying to insert something right before
position 4 of a 2-element list, what should happen?
Should an error be signaled?
Should the list be filled up with dummy elements to make it possible?
Similarly, we have not specified what the legal ranges are of the
\code{position} variable for the other operations.
What we intended was of course that \code{position} must be between 0
and \code{size-1}
(where \code{size} is the number of elements in the list)
for all operations except \code{insert},
where it needs to be between 0 and \code{size}.

Of course, we could add other functions that should be supported,
such as \code{size()} or \code{isEmpty()}.
The textbook lists several others,
but the four we have given are really the key functions that most
define a \code{List}.

You are probably already familiar with the C++ \code{vector} and
\code{deque} data types, which are two natural implementations of the
\code{List} data type (and provide a few other functions).
In fact, we will see how the \code{vector} implementation works
internally in Chapter~\ref{chap:arraylists}.
But \code{vector} and \code{deque} are not the only possible implementations,
and we discuss implementations based on linked lists as well.

\item[Bag (set):] 
A \code{set} (called \code{Bag} in the textbook) supports the
following operations:
\begin{enumerate}
\item \code{add (T item)}: Adds the item into the set.
\item \code{remove (T item)}: Removes the item from the set.
\item \code{bool contains (T item)}: Returns whether the set contains
  the given item.
\end{enumerate}

This is a rudimentary implementation of a mathematical set (see CSCI 170).
Again, our specification is not complete.
For instance, we do not say what should happen if an item is added multiple times:
are multiple copies added, or only one, and is an error signaled?
If multiple copies are added, then what does \code{remove} do?
Does it remove just one of them or multiple?

Typically, in a \code{set}, we only allow one copy.
If we want to allow multiple copies of the same item,
we call it a \todef{multiset}.
This convention is pretty widely used, but not universally.

\item[Dictionary (map):]
A \code{map} (or \code{Dictionary}, as it is called in the textbook)
is a data structure for creating and querying associations between
\todef{keys} and \todef{values}.
Both keys and values can be arbitrary data types \code{keyType} and
\code{valueType} themselves.
The supported operations are: 

\begin{enumerate}
\item \code{add (keyType key, valueType value)}:
  Adds a mapping from \code{key} to \code{value}.
\item \code{remove (keyType key)}:
  Removes the mapping for \code{key}.
\item \code{valueType get (keyType key)}:
  Returns the value that \code{key} maps to.
\end{enumerate}

Again, this is a bit underspecified.
What happens for \code{remove} or \code{get} if the given \code{key} is not in the map? 
Should they signal an error?
Also, if another mapping is added for the same \code{key}, what will
happen? Can two mappings co-exist? Will an error be signaled? Does the
new one overwrite the old one?
If there are two mappings for the same key, then which one does
\code{get} use to determine what to return?

Typically, one would not allow multiple mappings for the same key.
Again, this is widely, but not universally, accepted as standard.
Some people call a \code{map} that allows multiple mapping per key a
\code{multi-map}.

If you want to think about it mathematically, 
the \code{map} data type implements basically a (partial) function from
keys to values.
In a function (partial or not), you cannot have a key mapping to
multiple values.
\end{description}

Let us look a bit at commonalities and differences between these
abstract data types. 
First, all of them support storing data and accessing them;
after all, they are abstract data types.

The big difference between \code{List} and the others is that a list
really cares about the order. 
A list in which the number 2 is in position 0 and the number 5 in
position 1 is different from a list in which the two numbers are in
opposite order.
On the other hand, neither a \code{set} nor a \code{map} care about
order. An element is either in a set or not, but there is not even any
meaning to such a thing as an index in a set. 
Similarly, elements map to others under a \code{map}, but there is no
sense in which the elements are ordered or in a designated position.

Let us illustrate this in terms of some applications for these data
types. A \code{List} may be a good thing for a music playlist,
the lines in a computer program, or the pages of a book.
For all of them, the order of the items is important, as is the
ability to access specific items by index (a page, or a place in a
playlist).
On the other hand, a \code{map} is a natural fit for any kind of
directory (student records or e-mail address, phone book, dictionary
of words, web page lookup by search query).
Here, it does not matter in what order the items are stored,
so long as it is easy, given the key,
to find the corresponding record.

While a \code{List} is thus fundamentally different from a \code{map},
a \code{set} is actually quite similar to a \code{map}. The main
difference is that for a \code{map}, we store additional information
for each included item, whereas for the \code{set},
we only remember the presence (or absence) of items. 
In other words, \code{set} is a special case of \code{map}, where the
value associated with each key is some kind of dummy that is never
used. For that reason, we will not really look much at implementing
\code{set} (except in Section~\ref{sec:bloom-filters}). 
Instead, we will put a lot of emphasis on implementing \code{map},
and we then typically get a \code{set} implementation for free.

Of course, just because a \code{List} by itself is a very different
object from a \code{map} does not mean we cannot use one to implement
the other. In fact, a pretty natural (though not very efficient) way
of implementing maps is to use lists as the place to store the data.
It is important to notice, though, that this is now a ``how'' question:
just because we can use one technology to solve a different problem
does not mean that the two are the same.

Learning which ADT most naturally fits the requirements of what you
are trying to do is an important skill.
That way, you separate out the implementation of your code that uses
the ADT from the implementation of the ADT itself,
and structure your code much better. 