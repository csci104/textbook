The ADTs we have learned about and implemented so far, and
most of the ones we will learn about later, actually come
pre-implemented with C++, in the Standard Template Library (STL).
The STL also contains implementations of several frequently used
algorithms. 

While many students surely have been waiting eagerly to use STL
implementations instead of having to code data types from scratch,
some may wonder why one would use STL. The advantage of standard
libraries is that they save us some coding, and more importantly, a
lot of debugging, as they presumably have been much more carefully
debugged than our own code. When our own project is big enough, we'd
like to avoid having to debug the basic data structures \emph{in
  addition to} the high-level logic.
The contents of STL can be roughly categorized as follows:

\begin{description}
\item[Containers]: These are data structures; the name arises because
  they \emph{contain} items. They can be further divided as follows:
\begin{enumerate}
\item Sequence containers: These are data structures in which items
  are accessed by the position they are stored at, like arrays.
  Our example from class was called \code{List}.
\item Associative containers: Here, items are accessed via their
  ``identity'' or key. The user does not know which index an element
  is stored at --- in fact, there may not be a clear notion of an index.
  The associative data structures we have seen are \code{Set} and
  \code{Map}.
\item Adapters: these are restricted interfaces that give you specific
  functionality, such as \code{Stack}, \code{Queue}, or \code{Priority
    Queue} (we'll learn about that one soon). 
  They are often implemented on top of other classes, such 
    as \code{List}.
\end{enumerate}
		
\item[Algorithms]: To use pre-implemented algorithms, you should
\code{\#include<algorithm>}. The algorithms all live in the 
\code{namespace std}. They are roughly divided as follows:
\begin{enumerate}
\item Search and compare: these are mostly iterator-based. They allow
  you to run some algorithm (compare elements to a given one, count/add all
  elements, etc.) on all elements of a container.
  To use these algorithms, you need to pass in an iterator of the
  container. A brief introduction to iterators is given below, and we
  will discuss them in more detail later in the semester, and in
  Chapter~\ref{chap:iterators}. 
%  Referring back to the discussion of the Map-Reduce paradigm we
%  talked about in the context of iterators, these are typical
%  functions that would contribute the ``Reduce'' side.
\item Sequence modification: These will affect the entire container,
  such as filling an array with zeros, or performing a computation on
  all elements. They are also iterator-based. 
%  In the context of the Map-Reduce paradigm, these algorithms would
%  fit with the ``Map'' part.
\item Miscellaneous: These contain mostly sorting algorithms, and
  algorithms for manipulating heaps, but also computing moments of a
  distribution and partitioning items. 
\end{enumerate}
\end{description}

\section{More Details on Container Classes}

All containers implement at least the following functions:
\begin{verbatim}
bool empty();
unsigned int size();
operator= is overloaded
iterators
\end{verbatim}
	
The \code{size()} function doesn't tell you how many elements are in
the container, but rather how many elements it can hold (which we
typically call its \emph{capacity} in class). 
	
For all of the container classes described next, if you want to use
them, you need to \code{\#include<container-name>}, where
\code{container-name} is the name of the container you wish to use.
You also must either use \code{namespace std} or precede each
container name with \code{std::}. Remember that except in your main
file, using \code{namespace std} is generally a bad idea, as it forces
every program that includes your code to use that namespace, sometimes
causing clashes in variable or class names.
	
Also, as the name says, these classes are ``templated.'' 
This means that you can easily build stacks, queueues, maps, etc., with
any kinds of data stored in them (whereas earlier, we just built them
for integers). You will learn later, in Chapter~\ref{chap:templates},
how to build your own template classes. (It's not difficult, but can
cause compiler/linker errors if you're not careful.) For now, you need
to know how to use them. That's actually not hard. 
When STL has declarations such as

\begin{verbatim}
template <class T>
class stack { 
  ...
  void push (T item);
  ...
};

template <class keyType, class valueType>
class map { 
  ...
  void erase (keyType key);
  ...
}
\end{verbatim}

you can define your own versions as follows:
\begin{verbatim}
stack<string> myStringStack;
map<int,bool> myIntToBoolMap;
\end{verbatim}

You can substitute any types there, including ones you have defined
yourself, or other container types. (For instance, you could have a
map each of whose values is a stack.) To use the functions, you can
now write

\begin{verbatim}
string s = "Hello World";

myStringStack.push (s);
myIntToBoolMap.erase (5);
\end{verbatim}

In other words, C++ will automatically make sure that the right
functions exist for the types you specified.

\subsection{Sequence Containers}
As we mentioned above, sequence containers are
similar to \code{List<T>} from class, in that they implement access to
elements typically by index position.
They allow reading, overwriting, inserting, and removing. 
Also, most of them implement swapping, and several other functions.
There are several sequence container classes, with different tradeoffs
and implementations.

\begin{description}
\item[\code{vector<T>}]: This is basically our \code{List<T>} type,
implemented using an expanding (doubling) array internally.

\item[\code{array<T>}]: This implements a fixed-size array,
i.e., it does not grow and does not provide \code{insert} or \code{remove}. 
Its interface and functionality is basically the same as a standard array.
(It is available only starting with C++11.)

The main advantage over \code{T* a = new T[100]} is that it catches
array indices out of bounds and throws exceptions, instead of leading
to segfaults or other bad errors.
			
\item[\code{list<T>}]: This is a doubly-linked list. It does not allow
access by index. 
In C++11, there is also a singly linked list as \code{forwardlist<T>};
it saves a bit of memory.
The linked list contains some additional list-specific operations such
as combining and splicing lists.
			
\item[\code{deque<T>}]: This is \emph{not} a \code{deque} in the
sense that it combines a \code{stack} with a \code{queue}.
Rather, it is a data type similar to \code{vector<T>}, but implemented
with non-contiguous memory blocks. (Basically, it internally keeps a
lookup table which of multiple separate memory blocks to access when a
particular index is requested.)

Because it does not need to copy over all elements when the size
grows, inserting elements is faster.
On the other hand, some operations have more complex implementations
and are a bit slower. 
If you anticipate that you will be spending a lot of time growing your
arrays, consider using \code{deque<T>} instead of \code{vector<T>}.
\end{description}
		
\subsection{Adapter Containers}
The adapter containers are called this because they ``adapt'' the
interface of another class, and are not implemented from scratch.
They provide special and restricted functionality.
The three adapters in STL are \code{stack<T>}, \code{queue<T>}, and 
\code{priority-queue<T>}, implementing the data types we have learned
about in this class, or (in the case of \code{priority-queue<T>}) will
learn about soon.

\subsection{Associative Containers}
In sequence containers, we access elements by their position in the
container, so it matters \emph{where exactly} an element is stored.
In associative containers, we only know that we want to insert,
remove, and look up elements, but we don't care where or how they are
stored, so long as the data structure can find them for us (by their
identity or key) when we want to access them.

The two associative container types we have seen so far are
\code{Set<T>} and \code{Map<T1, T2>}. Recall that \code{Set<T>}
provides the following functions:
	
\begin{verbatim}
   insert(T item)
   remove(T item)
   bool contains(T item)
\end{verbatim}
			
A \emph{Map/Dictionary} is similar, but it creates an association between
a key (such as a word) and a value (such as the definition of the word).
The name used in STL and elsewhere is usually \code{map}.
The functions that a map should provide are:
			
\begin{verbatim}
   insert (keyType & key, valueType & value)
   remove (keyType & key)
   valueType & lookUp (keyType & key)
\end{verbatim}
			
While this is how we have been thinking about sets and maps, the
actual implementation and naming in STL is a little different, as
described below.

\subsubsection{Different Set/Map classes}
STL provides implementations of \code{set} and \code{map}, differing
in several parameters.
\begin{description}				
\item[\code{set<T>}]: implements a set that can contain at most one
  copy of each element, i.e., no two elements may have the same key.
\item[\code{multiset<T>}]: implements a multiset of elements, which
  means that it may contain multiple elements with the same key.
\item[\code{map<keyType, valueType>}]: implements a map from keys to values, ensuring
  that there is at most one copy of each key.
\item[\code{multimap<keyType, valueType}]: implements a map from keys to values that
  allows the data structure to contain more than one copy of a key.
\end{description}

All of these data types are implemented using balanced search trees
(which we will learn about in a month or so). 
What this means is that you will have to provide a comparison operator
(i.e., overload \code{<=}, \code{<} etc.)
for your type \code{T} if you want to use these types, or hope that
whatever default is implemented will work well enough.

There is an alternative implementation using hash tables (which we
will learn about right after balanced trees). 
To access those implementations instead, you put the word 
\code{unordered\_} in front of the type's name, e.g.,
\code{unordered\_{}map}.

The hash table implementation is often faster, but it depends on a
good implementation of a hash function, which you may have to write
and provide yourself to suit your type \code{T}.
On the other hand, the tree-based version is very fast at iterating
over all entries in sorted order, since that's how they are stored
internally anyway.

\subsubsection{Functions in STL Set/Map classes}
Here are the most relevant functions for set in STL:

\begin{description}
\item[\code{set::insert (T element)}]: inserts the element into the set. It
  returns an iterator to the element, but this is usually not so
  relevant for you.
\item[\code{void set::erase (T element)}]: removes the element from the set.
\item[\code{size\_type set::count (T element)}]: returns the number of times
  the element occurs in the set, which is 0 or 1 for a regular set
  (not multiset). Allows you to test if the set contains the element.
\item[\code{iterator set::find (T element)}]: returns an iterator to the
  element in the set. You can use this for removing the element. If
  the element is not in the set \code{s}, the iterator will be equal
  to \code{s.end()}; see the discussion of iterators below.
\end{description}

For maps, while we had written \code{add} as a function with two parameters,
the way it is actually implemented in STL is with just a single
parameter, which is of type \code{pair}; a \code{pair} is just a
\code{struct} of two individual types. It is essentially defined as follows:

\begin{verbatim}
template <class S, class T>
class pair {
  public:
  S first;
  T second;

  Pair (S s, T t) { first = s; second = t; }
}
\end{verbatim}

This type is used in a number of places, as we will see. Throughout
the following list, we assume that we have a \code{map<keyType, valueType>}.

\begin{description}
\item[\code{map::insert (pair<keyType,valueType>)}]: inserts the association
  between \code{p.first} and \code{p.second} into the map, and returns
  an iterator to that element. If the map already contained an
  association for the key \code{p.first}, \code{insert} does not
  change anything, and instead returns an iterator to the existing pair.
\item[\code{void map::erase (keyType key)}]: removes the association for
  \code{key} from the map. If there was no association for the key, it
  doesn't do anything.
\item[\code{size\_type map::count (keyType key)}]: returns the number of
  associations for the given \code{key}, which is 0 or 1 for a regular
  map (not multimap). Allows you to test if the map already has an
  assignment for the key.
\item[\code{iterator map::find (keyType key)}]: returns an iterator to the
  pair \code{(key, value)} associated with \code{key} in the map. 
  If the key has no association in the map \code{m}, the iterator will
  be equal to \code{m.end()}; see the discussion of iterators below.
  The iterator iterates over \emph{pairs}, so if you want to extract
  the value, you get it with \code{m.find(key)->second} (and the
  key itself should be \code{m.find(key)->first}). Make sure to test
  that there was actually a mapping for your \code{key} before you
  dereference the iterator.
\item[\code{valueType \& map::operator[] (keyType key)}]: allows you to
  treat your map essentially like an array, by writing things like
\begin{verbatim}
map<string, int> myMap;
myMap["Hello World"] = 101;
myMap["Hello World"] = 103;
myMap["Thanks for the fish!"] = 42;
cout << myMap["Hello World"];
\end{verbatim}
  Different from the \code{insert} function, this one allows you to
  overwrite the mapping for a given key, e.g., the code above would
  output \code{103}. When you try to read the value for a key that
  isn't in the map, this will \emph{insert} the key into the map, with
  some dummy value assigned to it. It will then return the dummy
  value. So be forewarned in using this!
\end{description}

\subsection{Iterators}
Iterators are a systematic and unified way of traversing arrays and
other data structures. You will learn much more about them (including
how to write your own) in Chapter~\ref{chap:iterators}. For now, just
a quick primer on how to use them.

When you go through an array with a loop, your code might look like
the following:

\begin{verbatim}
int a[10];
int *i;

for (i = a; i < a + 10; ++ i) 
   cout << (*i);
\end{verbatim}

Notice that instead of writing \code{a[i]}, we basically treated the
index itself as a pointer. Iterators try to resemble this code as much
as possible. When you have a data structure that implements/allows
iterators, it will give you two functions: \code{a.begin()} and
\code{a.end()}. The former gives you an iterator pointing to the
``first'' element, and the latter an iterator pointing to the element
just after you are done (which in the above loop is \code{a+10}). So
you would write code like this:

\begin{verbatim}
MyDataStructure a;
iterator i;

for (i = a.begin(); i != a.end(); ++ i)
   cout << (*i);
\end{verbatim}

Iterators typically overload \code{=}, \code{==}, \code{!=},
\code{++}, \code{*}, and \code{->} so that your code looks as similar
as possible to the loop over arrays. It is convention that
\code{a.end()} is used for ``We're done''.

Therefore, when \code{it = map::find(...)} returns an iterator, its value will
equal \code{m.end()} if the element was not found, and otherwise, it
can be treated almost like a pointer in an array. In particular, you
can use \code{*it} to get to the \emph{pair} that the iterators is
pointing to, and \code{(*it).second} or \code{it->second} to reference
its second component, which is the value you want.

You can also output all the values in a map with an iterator, as
follows:
\begin{verbatim}
map<string, int> m;
// put stuff into the map here first

for (iterator it = m.begin(); it != m.end(); ++it)
   cout << "Key: " << it->first << "  Value: " << it->second << "\n";
\end{verbatim}


