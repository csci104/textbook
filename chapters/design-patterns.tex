Throughout the semester, as you have learned about object-oriented
design, we have emphasized that its main benefit is the ability to
cleanly reuse code later, by separating out functionalities and
putting them where they belong, and by isolating interfaces
(\emph{what} functions a class provides) from implementations
(\emph{how} they are performed). Along the way, in addition to
inheritance as a natural extension mechanism, we have also seen more
interesting ways in which objects from different classes can interact,
each taking specific roles and responsibilities within a larger task,
and interacting with each other in pre-specified ways.

Perhaps the two most non-trivial ones were the \todef{Iterator}
paradigm (discussed in Chapter~\ref{chap:iterators}) and the
interaction between slots and signals in Qt in Chapter~\ref{chap:qt}.
These are examples of \todef{design patterns}: ways in which objects
can interact with each other that occur with at least some regularity,
and which --- when considered in the design of your class structure
--- will make it easier later to extend and reuse your code.
Design patterns are one of many important aspects of the broad field
of Software Engineering, and you will learn about them a little more,
and Software Engineering a lot more, whenever you take CSCI 201 and,
more importantly, the Software Engineering class CSCI 310.
For now, to show you a little more how to think about classes and
their interactions, we will discuss a few common design patterns.
Some of them are obvious, some are pretty clever, some are just
workarounds for weaknesses in the C++ or Java programming
languages. There are many more than we discuss here --- this is just
to give you a little taste.
For the student interested in learning more, we recommend the classic
book on ``Design Patterns'' by Gamma, Helm, Johnson, and Vlissides, as
well as many online sources.

\section{Observer}
The \todef{observer} paradigm is something that we already saw in
action when we talked about Qt, and how slots in one class correspond
to signals in another. More generally, the observer design pattern
applies when you have one object that has some state (which may
change, e.g., because the user enters values), and other objects which
need to be kept informed about any changes in the first object, e.g.,
because they display the content of the object.

The classic example is something like a table of data as the first
object, and several ways of displaying it. For instance, you may have
a table of student grades, and display it as a histogram, pie chart,
and in table format. How many and which kinds of displays are open
will depend on the user's actions, so you can't predict ahead of time
how many observers the object will have. In fact, you shouldn't
hardwire too much what possible displays there are: perhaps in version
3 of your software, you'll also want to introduce 3-D charts and a
display with a function plot. At that point, you'd like to simply say
that there are now additional ways of displaying the data, without
having to change the class for the actual table.

The way to separate out the functionality of storing the data and
displaying it, and allow the easy creation of new ways of displaying,
is by providing particular interfaces:

\begin{enumerate}
\item The class for your data table has a way for observers to
  register themselves, such as \code{void Table::register (Observer *obs)}.
  When this is called, the table adds \code{obs} to a list of
  registered observers.
\item The class for the displays has a way to be notified of changes,
  such as \code{void Observer::notify (Table *tab)}.
  Whenever anything changes in the table, it calls the \code{notify}
  function for all its observers, passing itself as an argument.
  Then, the observers will know how to update the way they display (or
  otherwise interact with) the data from the table.
  Of course, the \code{Table} class will have to provide an interface
  for the observer to actually get the data.
\end{enumerate}

When this design pattern is followed, you can later add different ways
to display the data, without ever changing a line of code in the
\code{Table} class. You simply write a new class inheriting from
\code{Observer}, and implementing the \code{notify} function
accurately.

This paradigm is also called \todef{publish-subscribe}: one object
provides an interface for subscribing to updates, and the other has an
interface for being notified of updates. When you subscribe to a
mailing list that informs you of the upcoming concerts of your
favorite band, you follow the same paradigm: the band doesn't need
to know ahead of time who will be the subscribers: all it does is provide a
way to put yourself on the mailing list. In turn, you provide an
interface by giving an e-mail address, which is a standardized way for
the band to update you on its status changes. The way you process
those status updates (buy concert tickets, visit the web site, tell
your friends, be disappointed because you can't go, etc.) is then up to you.

\section{Visitor}
Suppose that you have written a parser for mathematical formulas (as
you did, for instance, on a homework). The best way to represent a
parsed formula is via its \todef{parse tree}. For instance, the parse
tree for the formula ``$2*(x+4)$'' would be represented as in
Figure~\ref{fig:parse-tree}. 

\begin{figure}[htb]
\begin{center}
\psset{unit=0.7cm,levelsep=7ex,tpos=0.4,arrowsize=0.1 3}
\begin{pspicture}(0,-4)(0,0.5)
\pstree{\Tcircle{*}}
        {
	 \Tcircle{2}
         \pstree{\Tcircle{+}}
                {
                 \Tcircle{x}
                 \Tcircle{4}
                }
        }
\end{pspicture}
\caption{The parse tree for ``$2*(x+4)$''. \label{fig:parse-tree}}
\end{center}
\end{figure}

Once you have this tree representation, a lot of things can be done
quite easily using DFS tree traversal (pre-order, in-order, or
post-order), such as:
\begin{itemize}
\item Printing the expression in a nicely formatted way.
\item Evaluating the expression at a particular value of $x$ (which in
  turn could be used to plot the function).
\item Substituting a different expression for $x$, e.g., $x=y^2+3$.
\end{itemize}
You can likely come up with quite a few others. 

The question is where the functions \code{string prettyPrint ()},
\code{double evaluate (double x)} or \code{ParseTree *substitute (char
  variableName, ParseTree *expressionToSubstitute)} should reside.
Clearly, the most natural fit is as member function of the class
\code{ParseTree}. But that doesn't feel quite right. 
For starters, the tree should mostly store the data, but it's not
really its job to print it. More severely, what if another user wants
to print the tree in a different format? By now, you know that you
wouldn't want that person to have to go in and change your code for
printing. 

The only other apparent solution is to give global access of all
relevant internals in your \code{ParseTree} to the outside world,
i.e., make almost everything global. But that is bad coding style, as
it reveals too much about the implementation, and doesn't allow you to
later rename variables or similar things.

Upon some reflection, this conundrum looks quite similar to what we
solved by using the \todef{Iterator} design pattern for container
classes (Chapter~\ref{chap:iterators}). 
It allowed us to access the elements of the container in a
structured and limited fashion. But upon further reflection, an
iterator is not quite enough here. 
Iterators simply spit out the elements of the container in some
pre-defined order. When we want to pretty-print the expression, we
need to see elements in an in-order traversal. When we want to
evaluate it, we need to traverse the tree post-order, and more
importantly, know about the structure.
And when substituting an expression, we actually need to change the
tree structure.

So whatever we come up with needs to allow more access than an
iterator. The \todef{Visitor} paradigm provides a particular interface
with which other objects can access the contents of nodes of the tree,
such as following the left/right subtrees, or learning (and possibly
replacing) what is stored at a node. This is like a compromise: the
contents aren't really global, but rather, the way in which other
objects can see what's in the tree and interact with it is carefully
planned out. 

One can conform to these interfaces and at the same time write a large
number of different visitors which perform different functions.
Importantly, one can later change what exactly visitors do, or write
new ones, without changing the underlying \code{ParseTree} class.

\section{Factory}
Suppose that you are writing an office suite, which will comprise a
text editor, spread sheet, database, graphics program, and a few
others. You will want many of the user interface components to be the
same, both because you want the look-and-feel to be integrated for
your users, and because you want to reuse code and save some work.
One of the dialogs that you'll surely want is to create a new
document. Since this may entail quite a bit (asking the user,
generating a default file, opening it, etc.), you may intend to put
all of this into an object in charge of creating the document.
The problem is that the \code{generateDocument} function may have to
return different types of documents, based on what you are currently
creating: a new \code{TextDocument}, \code{SpreadsheetDocument},
\code{DatabaseDocument} or \code{GraphicsDocument}. So there will
definitely be differences between the classes you use for generating a
document, yet perhaps 90\% will be the same.

The solution is to use the \todef{Factory} design pattern. You define
an abstract class \code{Document}, from which all your types of
document will inherit. You also define an abstract class
\code{Factory}, which implements all of the common functions, and has
a virtual function \code{Document *generateDocument ()}.
Then, you can inherit from \code{Factory} to produce new classes
\code{TextFactory}, \code{SpreadsheetFactory}, \code{DatabaseFactory},
and \code{GraphicsFactory}. Each overwrites the virtual function
\code{generateDocument ()} to return the specific type of document you
want in this instance. For instance, you can now write the following
code:

\begin{verbatim}
Factory *factory = new TextFactory ();
Document *doc = factory->generateDocument ();
\end{verbatim}

This will ensure that \code{doc} contains (or rather, points to) a
\code{TextDocument} object afterwards. As you can see, you will need
to change very little in order to make this generate
\code{SpreadsheetDocument} objects instead.

So to summarize, the idea of the \todef{Factory} design pattern is that
you have an abstract \code{Factory} class, whose job it is to produce
objects that your program needs. By inheriting from this abstract
class, you produce different concrete factories, each of which produce
the objects of the specific type you need. All of the specific types
are subtypes of the abstract type that your abstract factory
produces. This way, the factories can reuse a lot of code, even though
they produce different types of objects. It also makes the code very
extensible: if later on, you decide that your office suite will have a
sound processing component, you can just define one new class of
\code{Factory}, and leave almost everything else the same.

\section{Adapter/Wrapper}
Suppose that you were working on a project with a friend, and needed a
Hashtable programmed from scratch. You asked your friend to have a
function \code{insert(const KeyType \& key, const ValueType \& value)},
but unfortunately, he wasn't paying attention, and instead called the
function \code{add (const Pair<KeyType, ValueType> \& toAdd)}.
In your code, you have a lot of references to \code{insert}, but your
friend has also used his class extensively already, and has a lot of
references to \code{add}. What to do?

Because there are many references to \code{add} in your friend's code,
you can't just change the name of the function. And if you've written
enough code (or there are other people on the team who were also
relying on this specification), you shouldn't just replace all
occurrences. A simple and obvious solution is to use an
\todef{adapter}: a class that provides the user interface you wanted,
and internally just translates it to what you have. Here is how you
could do that:

\begin{verbatim}
class HashtableAdapter : private FriendsHashtable { 
   // use private/protected inheritance or member element of type FriendsHashtable

  void insert (const KeyType & key, const ValueType & value)
     { this->add (Pair (key, value)); }

  // identical copies of other functions
};
\end{verbatim}

By doing this, you avoid having to rewrite everything, and can reuse
your friend's code as though it were using the interface you actually
need.

Remember that you saw the word \emph{adapter} before? When we talked
about C++ STL classes implementing standard data structures
(Chapter~\ref{chap:STL}), we mentioned that it has a few adapter
classes, \code{Queue} and \code{Stack}. They are called that because
they just put a different interface on the \code{vector} class. 
Instead of writing \code{v.insert (v.size(),item)}, you can
write \code{v.push (item)}. So in that sense, STL considers
\code{queue} and \code{stack} simply adapters put on top of the
\code{vector} class. Of course, from a data structures viewpoint, they
are very different, but in terms of implementation, they follow the
\todef{Adapter} design pattern.