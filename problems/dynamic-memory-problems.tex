\section{Exercises}

\subsection{\ReviewQuestions}

\begin{exercise}
  What is the difference between stack and heap memory?
  As a general rule, what is stored on the stack and what is stored on
  the heap?
\end{exercise}

\begin{exercise}
  When you have two variables of the same name \code{x} in a block of code,
  which one is accessed when you refer to \code{x}?
\end{exercise}

\begin{exercise}
  Describe what the operators \code{*} and \code{&} do.
\end{exercise}

\begin{exercise}
  Why can the compiler not always determine all memory requirements at
  compile time?
\end{exercise}

\begin{exercise}
  Besides your program's variables/data, what else is stored in
  memory on your computer? What is stored in the stack memory when
  your program runs (besides local variables)?
\end{exercise}

\begin{exercise}
  What could happen when you try to read/write a memory location that
  does not belong to your program?
\end{exercise}

\begin{exercise}
  Why is it a good idea to initialize all your pointers to
  \code{nullptr} when you declare them
  (unless you already know what value they need to take)?
\end{exercise}

\begin{exercise}
  What is it called when you dynamically allocate memory but do not
  deallocate it? Why can this be a problem?
\end{exercise}

\begin{exercise}
  Why should you not assume a fixed number of bits/bytes for the
  length of an integer or double?
\end{exercise}

\begin{exercise}
  Why should you always set a pointer to \code{nullptr} after
  deallocating its memory?
\end{exercise}

\subsection{\EasyQuestions}

\begin{exercise}
  What is the difference between \code{int *p = new int (5);}
  and \code{int *p = new int [5];}?
\end{exercise}

\begin{exercise}
  Suppose that you have a variable \code{int *p}, which happens to be
  stored in location 100.
  Which memory location is accessed when you write
  \code{p+1}?
  How about when you write \code{(void*) p + 1}?
\end{exercise}

\begin{exercise}
  In the following code, will all the allocated memory be correctly
  deallocated? Why or why not?
\begin{verbatim}
  int *p = new int [50];
  for (int i = 0; i < 50; ++i)
     p[i] = i;
  p = nullptr;
\end{verbatim}
\end{exercise}

\subsection{\MediumQuestions}
\begin{exercise}
  Fill an array with numbers, read between array positions.
\end{exercise}

\subsection{\HarderQuestions}
\begin{exercise}
  Skyskraper
\end{exercise}