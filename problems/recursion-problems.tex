\section{Exercises}

\subsection{\ReviewQuestions}

\begin{exercise}
  What does the term ``recursive'' mean?
\end{exercise}

\begin{exercise}
  What is the defining property of a recursive function?
\end{exercise}

\begin{exercise}
  Why is it essential for a recursive function to have at least one
  base case?
\end{exercise}

\begin{exercise}
  Why is it essential that the recursive calls in a recursive function
  be made on smaller inputs?
\end{exercise}

\begin{exercise}
  What property makes it easy to replace a recursive function with a
  loop?
\end{exercise}

\begin{exercise}
  How does recursive exhaustive search work?
\end{exercise}

\begin{exercise}
  What is the difference between head recursion and tail recursion?
\end{exercise}

\begin{exercise}
  What is the difference between direct and indirect recursion?
\end{exercise}

\subsection{\EasyQuestions}
\begin{exercise}
  Write a recursive program using tail recursion
  that returns the sum of integers in a given array.
  Use the following function header:

\begin{verbatim}
  int tailSum (int* array, int n, int &tempSum);
\end{verbatim}
  Obviously, you should not use any loops anywhere to get something
  out of this exercise.
\end{exercise}

\begin{exercise}
  Write a recursive program using head recursion
  that returns the sum of integers in a given array.
  Use the following function header:

\begin{verbatim}
  int headSum (int* array, int n);
\end{verbatim}

  In particular, do not pass around any variable by reference here.
  (And of course, do not use any loops.)
\end{exercise}

\begin{exercise}
  Write a recursive program using tail recursion
  that returns the maximum of integers in a given array.
  Use the following function header:

\begin{verbatim}
  int tailMax (int* array, int n, int &tempSum);
\end{verbatim}
  Obviously, you should not use any loops anywhere to get something
  out of this exercise.
\end{exercise}

\begin{exercise}
  Write a recursive program using head recursion
  that returns the maximum of integers in a given array.
  Use the following function header:

\begin{verbatim}
  int headMax (int* array, int n);
\end{verbatim}

  In particular, do not pass around any variable by reference here.
  (And of course, do not use any loops.)
\end{exercise}

\begin{exercise}
  Write a recursive program using tail recursion
  that returns the concatenation of all strings in a given array.
  Use the following function header:

\begin{verbatim}
  string tailConcatenation (string* array, int n, string &tempString);
\end{verbatim}
  Obviously, you should not use any loops anywhere to get something
  out of this exercise.
\end{exercise}

\begin{exercise}
  Write a recursive program using head recursion
  that returns the concatenation of all strings in a given array.
  Use the following function header:

\begin{verbatim}
  string headConcatenation (string* array, int n);
\end{verbatim}

  In particular, do not pass around any variable by reference here.
  (And of course, do not use any loops.)
\end{exercise}

\begin{exercise}
  Write a recursive program that prints all numbers base 3 of a given
  length $n$.
\end{exercise}

\begin{exercise}
  Extend the recursive context-free grammar for arithmetic expressions
  to include subtraction and division.
\end{exercise}

\subsection{\MediumQuestions}
\begin{exercise}
  Write a recursive function that returns the input string in reverse
  order. (So ``Problem'' becomes ``melborP''.)
  Use the following function header:

\begin{verbatim}
  string reverseString (string* a);
\end{verbatim}

  You cannot use loops, nor should you add any arguments to the
  function.
  But you can use the \code{substr} function in the \code{string} class.
\end{exercise}

\begin{exercise}
  Write a recursive program that allows a user to enter a string
  \code{s} and a number \code{n}, and prints all strings of length $n$
  made up of characters occurring in \code{s}.
\end{exercise}

\begin{exercise}
  Write a recursive Ternary Search algorithm to find an integer in an
  array of integers.
  To do so, your algorithm should check the positions
  $\frac{1}{3}$ and $\frac{2}{3}$ through the (remaining) array to 
  determine which part the target number is in,
  then recurse on that part.
  Your solution should not contain any loops.
\end{exercise}

\begin{exercise}
  Give a recursive definition of a correct sequence of parentheses.
  Such a sequence can use the characters `[', `(', `)', `]',
  and parentheses need to match ``outside in.''
  Some examples that should fit in your definition are
  ``'' (empty string), ``([])[]((()))'', ``(((([[()]]))))''.
  Examples that should not be allowed under your definition are
  ``(()'' (unmatched), ``][' (wrong order), ``((])'' (mismatch).
\end{exercise}

\subsection{\HarderQuestions}
\begin{exercise}
  Write a recursive Backtracking program to color a
  graph.\footnote{Applications include classroom scheduling, map
    coloring below, and many others.}
  You are given an $n \times n$ array of 0/1 entries,
  where 1 in the $(i,j)$ position means that $i$ and $j$ are
  incompatible, and must be given different colors.
  (For instance, two classes have a common student/teacher,
  or two countries share a border, or two students do not get along
  and cannot be put on the same team.)
  Your goal is to find an assignment of ``colors'' (numbers) to the
  $n$ individual entries such that no two incompatible individuals
  have the same color.
  You want to do this with at most $k$ total colors.
\end{exercise}

\begin{exercise}
  Solve the previous coloring problem specifically for a map.
  Your input will now be a map, represented as a 2-D array of
  characters.
  All squares with the same character belong to the same country.
  Inputs are such that each country is \emph{connected}, meaning that
  you can go from any of its squares to any other going only
  up/down/left/right, without leaving the country.
  Two countries share a border if you can go from some square of one of
  them to some square of the other in one up/down/left/right step.
  If two countries share a border, they cannot be given the same color
  (because you could not tell them apart on a map).
  For a given map, find a coloring with at most 4 colors.
  (This is always possible by the famous Four-Color Theorem.)
\end{exercise}

\begin{exercise}
  Use a recursive Backtracking solution to solve the following
  problem of selecting a team for your startup company.
  You have $n$ candidate programmers,
  each with an ability score $a_i$ and a
  minimum salary $s_i$ you would need to pay them.
  You have a budget of $B$.
  The quality of your team is the sum of their ability scores.
  Compute the maximum quality you could obtain within your given
  budget.
  (If you think that Backtracking is not necessary, since you could
  just keep adding the best ``bang for the buck'' programmers you can
  afford, think about how your program would figure out to hire 10
  programmers with ability 1 and salary 1 using a budget of 10,
  instead of starting with a programmer with ability 5.9 and salary
  5.1.)
\end{exercise}